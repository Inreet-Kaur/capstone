\documentclass{article}

\usepackage{float}
\restylefloat{table}

\usepackage{booktabs}

\title{Team Contributions: POC\\\progname}

\author{\authname}

\date{}

%% Comments

\usepackage{color}

\newif\ifcomments\commentstrue %displays comments
%\newif\ifcomments\commentsfalse %so that comments do not display

\ifcomments
\newcommand{\authornote}[3]{\textcolor{#1}{[#3 ---#2]}}
\newcommand{\todo}[1]{\textcolor{red}{[TODO: #1]}}
\else
\newcommand{\authornote}[3]{}
\newcommand{\todo}[1]{}
\fi

\newcommand{\wss}[1]{\authornote{blue}{SS}{#1}} 
\newcommand{\plt}[1]{\authornote{magenta}{TPLT}{#1}} %For explanation of the template
\newcommand{\an}[1]{\authornote{cyan}{Author}{#1}}

%% Common Parts

\newcommand{\progname}{ProgName} % PUT YOUR PROGRAM NAME HERE
\newcommand{\authname}{Team \#, Team Name
\\ Student 1 name
\\ Student 2 name
\\ Student 3 name
\\ Student 4 name} % AUTHOR NAMES                  

\usepackage{hyperref}
    \hypersetup{colorlinks=true, linkcolor=blue, citecolor=blue, filecolor=blue,
                urlcolor=blue, unicode=false}
    \urlstyle{same}
                                


\begin{document}

\maketitle

This document summarizes the contributions of each team member up to the POC
Demo.  The time period of interest is the time between the beginning of the term
and the POC demo.

\section{Demo Plans}

% \wss{What will you be demonstrating}
Our project is a system which intends to speed up the documentation process for health care systems through audio transcription and automated report generation.\\
In the proof of concept demonstration we intend to demonstrate the core functionality of the system which is to take voice input and convert that into text and then classify it populate the a sample patient medical chart. This medical chart will contain fields like name, symptoms, next steps etc.\\
These two are the key features that are crucial to achieve the main goals of the project. 

In terms of the components that will be demonstrated, they go as follows:

\begin{itemize}
  \item \textbf{Frontend}: This will allow the user to record their voice, then based on the voice input it will populate a sample medical chart.
  \item \textbf{Voice to Text Module}: This will convert the voice input to text.
  \item \textbf{Classification Module}: This will classify which part of the text goes in which section of the medical chart.
\end{itemize}

\section{Team Meeting Attendance}

% \wss{For each team member how many team meetings have they attended over the
% time period of interest.  This number should be determined from the meeting
% issues in the team's repo.  The first entry in the table should be the total
% number of team meetings held by the team.}

\begin{table}[H]
\centering
\begin{tabular}{ll}
\toprule
\textbf{Student} & \textbf{Meetings}\\
\midrule
Total & 8\\
Gurleen Rahi & 8 \\
Inreet Kaur & 8 \\
Moamen Ahmed & 8 \\
Pranav Kalsi & 8 \\
\bottomrule
\end{tabular}
\end{table}

% \wss{If needed, an explanation for the counts can be provided here.}

\section{Supervisor/Stakeholder Meeting Attendance}

% \wss{For each team member how many supervisor/stakeholder team meetings have
% they attended over the time period of interest.  This number should be determined
% from the supervisor meeting issues in the team's repo.  The first entry in the
% table should be the total number of supervisor and team meetings held by the
% team.  If there is no supervisor, there will usually be meetings with
% stakeholders (potential users) that can serve a similar purpose.}

\begin{table}[H]
\centering
\begin{tabular}{ll}
\toprule
\textbf{Student} & \textbf{Meetings}\\
\midrule
Total & 2 \\
Gurleen Rahi & 2\\
Inreet Kaur & 1\\
Moamen Ahmed & 1\\
Pranav Kalsi & 1\\
\bottomrule
\end{tabular}
\end{table}

% \wss{If needed, an explanation for the counts can be provided here.}
We have recently found a new supervisor. Gurleen Rahi had a first in-person meeting to pitch our idea to the supervisor. We have now setup a biweekly check-in meeting in order to discuss our progress. We have also setup an in-person meeting at the clinic to get to know more about the environment we are dealing with and learn more about the current systems. 

\section{Lecture Attendance}

% \wss{For each team member how many lectures have they attended over the time
% period of interest.  This number should be determined from the lecture issues in
% the team's repo.  The first entry in the table should be the total number of
% lectures since the beginning of the term.}

\begin{table}[H]
\centering
\begin{tabular}{ll}
\toprule
\textbf{Student} & \textbf{Lectures}\\
\midrule
Total & 12\\
Gurleen Rahi & 9\\ 
Inreet Kaur & 9\\
Moamen Ahmed & 8\\
Pranav Kalsi & 10\\
\bottomrule
\end{tabular}
\end{table}

% \wss{If needed, an explanation for the lecture attendance can be provided here.}

\section{TA Document Discussion Attendance}

% \wss{For each team member how many of the informal document discussion meetings
% with the TA were attended over the time period of interest.}

\begin{table}[H]
\centering
\begin{tabular}{ll}
\toprule
\textbf{Student} & \textbf{Lectures}\\
\midrule
Total & 3\\
Gurleen Rahi & 3\\
Inreet Kaur & 3\\
Moamen Ahmed & 3\\
Pranav Kalsi & 3\\
\bottomrule
\end{tabular}
\end{table}

% \wss{If needed, an explanation for the attendance can be provided here.}

\section{Commits}

% \wss{For each team member how many commits to the main branch have been made
% over the time period of interest.  The total is the total number of commits for
% the entire team since the beginning of the term.  The percentage is the
% percentage of the total commits made by each team member.}

\begin{table}[H]
\centering
\begin{tabular}{lll}
\toprule
\textbf{Student} & \textbf{Commits} & \textbf{Percent}\\
\midrule
Total & 181 & 100\% \\
Gurleen Rahi & 48 & 26.5\% \\
Inreet Kaur & 44 & 24.3\% \\
Moamen Ahmed & 17 & 9.4\% \\
Pranav Kalsi & 72 & 39.8\% \\
\bottomrule
\end{tabular}
\end{table}

% \wss{If needed, an explanation for the counts can be provided here.  For
% instance, if a team member has more commits to unmerged branches, these numbers
% can be provided here.  If multiple people contribute to a commit, git allows for
% multi-author commits.}

\section{Issue Tracker}

% \wss{For each team member how many issues have they authored (including open and
% closed issues (O+C)) and how many have they been assigned (only counting closed
% issues (C only)) over the time period of interest.}

\begin{table}[H]
\centering
\begin{tabular}{lll}
\toprule
\textbf{Student} & \textbf{Authored (O+C)} & \textbf{Assigned (C only)}\\
\midrule
Gurleen Rahi & 28 & 35\\
Inreet Kaur & 22 & 34\\
Moamen Ahmed & 0 & 31\\
Pranav Kalsi & 25 & 40\\
\bottomrule
\end{tabular}
\end{table}

% \wss{If needed, an explanation for the counts can be provided here.}
Gurleen Rahi: Responsible for making issues for all the TA , supervisor,  and team meetings.\\
Pranav Kalsi: Responsible for making issues for the deliverables.\\
Inreet Kaur: Responsible for making issues for lectures and deliverables. \\
Gurleen Rahi, Pranav Kalsi, Inreet Kaur responsible for making issues for peer-reviews.\\
Moamen will be responsible for making issues for features and test suite.\\



\section{CICD}

% \wss{Say how CICD will be used in your project}

% \wss{If your team has additional metrics of productivity, please feel free to
% add them to this report.}

CI/CD will be critical for project development. We will be using \textbf{Jenkins} which is an open-source CI/CD development tool that will act as an automated DevOps tool.

Jenkins will allow us to achieve Continuous Integration and Continuous Delivery through the following:
\begin{itemize}
  \item \textbf{Continuous Integration} -- On the continuous integration time Jenkins offers automated build and testing. This will save a lot of overhead on testing and test feedback as Jenkins will be responsible for it not only will it build and test the code it will also offer instant feedback to the developer.
  \item \textbf{Continuous Deployment} -- Jenkins offers automated deployment functionalities meaning that once the build passes all tests it can be deployed into a production or pre-production environment. This will ensure the deployment is consistent, reliable, and efficient. This also will make sure that all features pass a minimum functionality standard ensuring that they are ready for production. Jenkins offers integrations with git as well so deployments will be version-controlled meaning they may be reverted as needed.
\end{itemize}

By implementing a CI/CD tool (Jenkins) we can ensure that code isn't riddled with errors and automate a lot of tasks which will increase productivity. Having Jenkins also will reduce the risk of human error in the project and will automate many areas of DevOps.


\end{document}