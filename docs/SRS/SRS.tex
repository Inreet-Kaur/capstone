% THIS DOCUMENT IS TAILORED TO REQUIREMENTS FOR SCIENTIFIC COMPUTING.  IT SHOULDN'T
% BE USED FOR NON-SCIENTIFIC COMPUTING PROJECTS
\documentclass[12pt]{article}

\usepackage{amsmath, mathtools}
\usepackage{amsfonts}
\usepackage{amssymb}
\usepackage{graphicx}
\usepackage{colortbl}
\usepackage{xr}
\usepackage{hyperref}
\usepackage{longtable}
\usepackage{xfrac}
\usepackage{tabularx}
\usepackage{float}
\usepackage{siunitx}
\usepackage{booktabs}
\usepackage{caption}
\usepackage{pdflscape}
\usepackage{afterpage}

\usepackage[round]{natbib}

%\usepackage{refcheck}

\hypersetup{
    bookmarks=true,         % show bookmarks bar?
      colorlinks=true,       % false: boxed links; true: colored links
    linkcolor=red,          % color of internal links (change box color with linkbordercolor)
    citecolor=green,        % color of links to bibliography
    filecolor=magenta,      % color of file links
    urlcolor=cyan           % color of external links
}

%% Comments

\usepackage{color}

\newif\ifcomments\commentstrue %displays comments
%\newif\ifcomments\commentsfalse %so that comments do not display

\ifcomments
\newcommand{\authornote}[3]{\textcolor{#1}{[#3 ---#2]}}
\newcommand{\todo}[1]{\textcolor{red}{[TODO: #1]}}
\else
\newcommand{\authornote}[3]{}
\newcommand{\todo}[1]{}
\fi

\newcommand{\wss}[1]{\authornote{blue}{SS}{#1}} 
\newcommand{\plt}[1]{\authornote{magenta}{TPLT}{#1}} %For explanation of the template
\newcommand{\an}[1]{\authornote{cyan}{Author}{#1}}

%% Common Parts

\newcommand{\progname}{ProgName} % PUT YOUR PROGRAM NAME HERE
\newcommand{\authname}{Team \#, Team Name
\\ Student 1 name
\\ Student 2 name
\\ Student 3 name
\\ Student 4 name} % AUTHOR NAMES                  

\usepackage{hyperref}
    \hypersetup{colorlinks=true, linkcolor=blue, citecolor=blue, filecolor=blue,
                urlcolor=blue, unicode=false}
    \urlstyle{same}
                                


% For easy change of table widths
\newcommand{\colZwidth}{1.0\textwidth}
\newcommand{\colAwidth}{0.13\textwidth}
\newcommand{\colBwidth}{0.82\textwidth}
\newcommand{\colCwidth}{0.1\textwidth}
\newcommand{\colDwidth}{0.05\textwidth}
\newcommand{\colEwidth}{0.8\textwidth}
\newcommand{\colFwidth}{0.17\textwidth}
\newcommand{\colGwidth}{0.5\textwidth}
\newcommand{\colHwidth}{0.28\textwidth}

% Used so that cross-references have a meaningful prefix
\newcounter{defnum} %Definition Number
\newcommand{\dthedefnum}{GD\thedefnum}
\newcommand{\dref}[1]{GD\ref{#1}}
\newcounter{datadefnum} %Datadefinition Number
\newcommand{\ddthedatadefnum}{DD\thedatadefnum}
\newcommand{\ddref}[1]{DD\ref{#1}}
\newcounter{theorynum} %Theory Number
\newcommand{\tthetheorynum}{TM\thetheorynum}
\newcommand{\tref}[1]{TM\ref{#1}}
\newcounter{tablenum} %Table Number
\newcommand{\tbthetablenum}{TB\thetablenum}
\newcommand{\tbref}[1]{TB\ref{#1}}
\newcounter{assumpnum} %Assumption Number
\newcommand{\atheassumpnum}{A\theassumpnum}
\newcommand{\aref}[1]{A\ref{#1}}
\newcounter{goalnum} %Goal Number
\newcommand{\gthegoalnum}{GS\thegoalnum}
\newcommand{\gsref}[1]{GS\ref{#1}}
\newcounter{instnum} %Instance Number
\newcommand{\itheinstnum}{IM\theinstnum}
\newcommand{\iref}[1]{IM\ref{#1}}
\newcounter{reqnum} %Requirement Number
\newcommand{\rthereqnum}{R\thereqnum}
\newcommand{\rref}[1]{R\ref{#1}}
\newcounter{nfrnum} %NFR Number
\newcommand{\rthenfrnum}{NFR\thenfrnum}
\newcommand{\nfrref}[1]{NFR\ref{#1}}
\newcounter{lcnum} %Likely change number
\newcommand{\lthelcnum}{LC\thelcnum}
\newcommand{\lcref}[1]{LC\ref{#1}}

\usepackage{fullpage}

\newcommand{\deftheory}[9][Not Applicable]
{
\newpage
\noindent \rule{\textwidth}{0.5mm}

\paragraph{RefName: } \textbf{#2} \phantomsection 
\label{#2}

\paragraph{Label:} #3

\noindent \rule{\textwidth}{0.5mm}

\paragraph{Equation:}

#4

\paragraph{Description:}

#5

\paragraph{Notes:}

#6

\paragraph{Source:}

#7

\paragraph{Ref.\ By:}

#8

\paragraph{Preconditions for \hyperref[#2]{#2}:}
\label{#2_precond}

#9

\paragraph{Derivation for \hyperref[#2]{#2}:}
\label{#2_deriv}

#1

\noindent \rule{\textwidth}{0.5mm}

}

\begin{document}

\title{Software Requirements Specification for \progname: subtitle describing software} 
\author{\authname}
\date{\today}
	
\maketitle

~\newpage

\pagenumbering{roman}

\tableofcontents

~\newpage

\section*{Revision History}

\begin{tabularx}{\textwidth}{p{3cm}p{2cm}X}
\toprule {\bf Date} & {\bf Version} & {\bf Notes}\\
\midrule
Date 1 & 1.0 & Notes\\
Date 2 & 1.1 & Notes\\
\bottomrule
\end{tabularx}

\newpage

\pagenumbering{arabic}


\section{Introduction}

\plt{The introduction section is written to introduce the problem.  It starts
  general and focuses on the problem domain. The general advice is to start with
a paragraph or two that describes the problem, followed by a ``roadmap''
paragraph.  A roadmap orients the reader by telling them what sub-sections to
expect in the Introduction section.}

\subsection{Purpose of Document}

\plt{This section summarizes the purpose of the SRS document.  It does not focus
  on the problem itself.  The problem is described in the ``Problem
  Description'' section (Section~\ref{Sec_pd}).  The purpose is for the document
  in the context of the project itself, not in the context of this course.
  Although the ``purpose'' of the document is to get a grade, you should not
  mention this.  Instead, ``fake it'' as if this is a real project.  The purpose
  section will be similar between projects.  The purpose of the document is the
  purpose of the SRS, including communication, planning for the design stage,
  etc.}

\subsection{Characteristics of Intended Reader} \label{sec_IntendedReader} 

The intended readers of this SRS document include project managers, software developers, testing engineers, and stakeholders directly involved in the design and implementation of the software system. Project managers and testing engineers would be directly involved in the development and testing process. They would generally have an education in computer science along with experience in software design, web or mobile development, and software testing. Project managers generally have experience in managing software projects and knowledge of software development processes. Stakeholders like doctors and nursing staff would have domain knowledge and could provide insights to understand the clinical workflow and documentation process. 

\subsection{Scope of Requirements} 

\plt{Modelling the real world requires simplification.  The full complexity of
  the actual physics, chemistry, biology is too much for existing models, and
  for existing computational solution techniques.  Rather than say what is in
  the scope, it is usually easier to say what is not.  You can think of it as
  the scope is initially everything, and then it is constrained to create the
  actual scope.  For instance, the problem can be restricted to 2 dimensions, or
  it can ignore the effect of temperature (or pressure) on the material
  properties, etc.}  

\plt{The scope section is related to the assumptions section
  (Section~\ref{sec_assumpt}).  However, the scope and the assumptions are not
  at the same level of abstraction.  The scope is at a high level.  The focus is
  on the ``big picture'' assumptions.  The assumptions section lists, and
  describes, all of the assumptions.}

\plt{The scope section is relevant for later determining typical values of inputs. The scope should make it clear what inputs are reasonable to expect. This is a distinction between scope and context (context is a later section).  Scope affects the inputs while context affects how the software will be used.}

\plt{The goal statements refine the ``Problem Description''
  (Section~\ref{Sec_pd}).  A goal is a functional objective the system under
  consideration should achieve. Goals provide criteria for sufficient
  completeness of a requirements specification and for requirements
  pertinence. Goals will be refined in Section “Instanced Models”
  (Section~\ref{sec_instance}). Large and complex goals should be decomposed
  into smaller sub-goals.  The goals are written abstractly, with a minimal
  amount of technical language.  They should be understandable by non-domain
  experts.}

\noindent Given the \plt{inputs}, the goal statements are:

\begin{itemize}
\item[GS\refstepcounter{goalnum}\thegoalnum \label{G_meaningfulLabel}:] \plt{One
    sentence description of the goal.  There may be more than one.  Each Goal
    should have a meaningful label.}
\end{itemize}

% References at the end of the document

\subsection{Organization of Document}

\plt{This section provides a roadmap of the SRS document.  It will help the
  reader orient themselves.  It will provide direction that will help them
  select which sections they want to read, and in what order.  This section will
  be similar between project.}

~\newpage

\section{System Description}

\subsection{Definitions, Acronyms, and Abbreviations}

\begin{tabular}{l l} 
  \toprule    
  \textbf{symbol} & \textbf{description}\\
  \midrule 
  A & Assumption\\
  GD & General Definition\\
  GS & Goal Statement\\
  LC & Likely Change\\
  R & Requirement\\
  SRS & Software Requirements Specification\\
  EHR & Electronic Healthcare record\\
  \bottomrule
\end{tabular}\\

The following is the glossary for this document:

\textbf{User} -- When referring to a user the document referes to a person whol will use this product

\subsection{System overview}

\subsection{System Functionality}

\subsection{User Characteristics} \label{SecUserCharacteristics}

The intended users of this product are healthcare professionals, specifically those involved in patient documentation, including doctors, nurses, and other clinical administrative staff. 

The following are the user characteristics associated with the intended users:

\textbf{Education Level} -- All users are expected to have a minimum of a college diploma and basic reading, writing, and speaking skills. The education level of the intended users will vary based on their role, generally a bachelor’s degree or college diploma in nursing, Medicine, or Healthcare fields. 

\textbf{Experience} -- All users will be expected to have basic familiarity with EHR systems or similar applications. This experience will depend on their roles ranging from entry-level positions to experienced professionals. 

\textbf{Technical Expertise} -- The users should have basic technical skills to use EHR systems and other healthcare software applications. In addition to this, the users should have familiarity with data entry processes and analytics interpretation.

\textbf{Accessibility Considerations} -- We anticipate that some users may have accessibility issues. The user interface should be designed for ease of navigation, incorporating various accessibility features.

Specific requirements must be specified to accommodate users with varying technical expertise and diverse educational backgrounds. This will ensure the application is usable by all healthcare staff, regardless of their abilities.


\subsection{System Constraints}

The following constraints will guide the system design and implementation:

\begin{itemize} 
  \item \textbf{System Compatibility} -- The system must be compatible with existing EHR systems for seamless integration and data exchange.

  \item \textbf{Compliance with Regulatory Standards} -- The system must adhere to healthcare regulations ensuring the confidentiality and security of patient data.

  \item \textbf{User Accessibility} -- The system must meet established accessibility standards to ensure usability for all healthcare staff, including those with disabilities.

  \item \textbf{Cost and Time} -- The system must be developed within the imposed budget and time restrictions.

\end{itemize}

\subsubsection{Assumptions and Dependencies} \label{sec_assumpt}

\subsubsection{Usage Scanerios}


~\newpage

\section{Requirements}

This section provides the functional requirements, the business tasks that the
software is expected to complete, and the nonfunctional requirements, the
qualities that the software is expected to exhibit.

\subsection{Functional Requirements}

\noindent \begin{itemize}

\item[R\refstepcounter{reqnum}\thereqnum \label{R_Inputs}:] \plt{Requirements
    for the inputs that are supplied by the user.  This information has to be
    explicit.}

\item[R\refstepcounter{reqnum}\thereqnum \label{R_OutputInputs}:] \plt{It isn't
    always required, but often echoing the inputs as part of the output is a
    good idea.}

\item[R\refstepcounter{reqnum}\thereqnum \label{R_Calculate}:] \plt{Calculation
    related requirements.}

\item[R\refstepcounter{reqnum}\thereqnum \label{R_VerifyOutput}:]
  \plt{Verification related requirements.}

\item[R\refstepcounter{reqnum}\thereqnum \label{R_Output}:] \plt{Output related
    requirements.}

\end{itemize}

\plt{Every IM should map to at least one requirement, but not every requirement
  has to map to a corresponding IM.}

\subsection{Nonfunctional Requirements}

\plt{List your nonfunctional requirements.  You may consider using a fit
  criterion to make them verifiable.}
\plt{The goal is for the nonfunctional requirements to be unambiguous, abstract
  and verifiable.  This isn't easy to show succinctly, so a good strategy may be
to give a ``high level'' view of the requirement, but allow for the details to
be covered in the Verification and Validation document.}
\plt{An absolute requirement on a quality of the system is rarely needed.  For
  instance, an accuracy of 0.0101 \% is likely fine, even if the requirement is
  for 0.01 \% accuracy.  Therefore, the emphasis will often be more on
  describing now well the quality is achieved, through experimentation, and
  possibly theory, rather than meeting some bar that was defined a priori.}
\plt{You do not need an entry for correctness in your NFRs.  The purpose of the
  SRS is to record the requirements that need to be satisfied for correctness.
  Any statement of correctness would just be redundant. Rather than discuss
  correctness, you can characterize how far away from the correct (true)
  solution you are allowed to be.  This is discussed under accuracy.}

\noindent \begin{itemize}

\item[NFR\refstepcounter{nfrnum}\thenfrnum \label{NFR_Accuracy}:]
  \textbf{Accuracy} \plt{Characterize the accuracy by giving the context/use for
    the software.  Maybe something like, ``The accuracy of the computed
    solutions should meet the level needed for $<$engineering or scientific
    application$>$.  The level of accuracy achieved by \progname{} shall be
    described following the procedure given in Section~X of the Verification and
    Validation Plan.''  A link to the VnV plan would be a nice extra.}

\item[NFR\refstepcounter{nfrnum}\thenfrnum \label{NFR_Usability}:] \textbf{Usability}
  \plt{Characterize the usability by giving the context/use for the software.
    You should likely reference the user characteristics section.  The level of
    usability achieved by the software shall be described following the
    procedure given in Section~X of the Verification and Validation Plan.  A
    link to the VnV plan would be a nice extra.}

\item[NFR\refstepcounter{nfrnum}\thenfrnum \label{NFR_Maintainability}:]
  \textbf{Maintainability} \plt{The effort required to make any of the likely
    changes listed for \progname{} should be less than FRACTION of the original
    development time.  FRACTION is then a symbolic constant that can be defined
    at the end of the report.}

\item[NFR\refstepcounter{nfrnum}\thenfrnum \label{NFR_Portability}:]
  \textbf{Portability} \plt{This NFR is easier to write than the others.  The
    systems that \progname{} should run on should be listed here.  When possible
    the specific versions of the potential operating environments should be
    given.  To make the NFR verifiable a statement could be made that the tests
    from a given section of the VnV plan can be successfully run on all of the
    possible operating environments.}

\item Other NFRs that might be discussed include verifiability,
  understandability and reusability.

\end{itemize}

\subsection{Rationale}

\plt{Provide a rationale for the decisions made in the documentation.  Rationale
should be provided for scope decisions, modelling decisions, assumptions and
typical values.}

\section{Likely Changes}    

\noindent \begin{itemize}

\item[LC\refstepcounter{lcnum}\thelcnum\label{LC_meaningfulLabel}:] \plt{Give
    the likely changes, with a reference to the related assumption (aref), as appropriate.}

\end{itemize}

\section{Unlikely Changes}    

\noindent \begin{itemize}

\item[LC\refstepcounter{lcnum}\thelcnum\label{LC_meaningfulLabel}:] \plt{Give
    the unlikely changes.  The design can assume that the changes listed will
    not occur.}

\end{itemize}

~\newpage

\section{References}

\newpage{}
\section*{Appendix --- Reflection}

\wss{Not required for CAS 741}

The information in this section will be used to evaluate the team members on the
graduate attribute of Lifelong Learning.  

The purpose of reflection questions is to give you a chance to assess your own
learning and that of your group as a whole, and to find ways to improve in the
future. Reflection is an important part of the learning process.  Reflection is
also an essential component of a successful software development process.  

Reflections are most interesting and useful when they're honest, even if the
stories they tell are imperfect. You will be marked based on your depth of
thought and analysis, and not based on the content of the reflections
themselves. Thus, for full marks we encourage you to answer openly and honestly
and to avoid simply writing ``what you think the evaluator wants to hear.''

Please answer the following questions.  Some questions can be answered on the
team level, but where appropriate, each team member should write their own
response:


\begin{enumerate}
  \item What went well while writing this deliverable? 
  \item What pain points did you experience during this deliverable, and how did
  you resolve them?
  \item How many of your requirements were inspired by speaking to your
  client(s) or their proxies (e.g. your peers, stakeholders, potential users)?
  \item Which of the courses you have taken, or are currently taking, will help
  your team to be successful with your capstone project.
  \item What knowledge and skills will the team collectively need to acquire to
  successfully complete this capstone project?  Examples of possible knowledge
  to acquire include domain specific knowledge from the domain of your
  application, or software engineering knowledge, mechatronics knowledge or
  computer science knowledge.  Skills may be related to technology, or writing,
  or presentation, or team management, etc.  You should look to identify at
  least one item for each team member.
  \item For each of the knowledge areas and skills identified in the previous
  question, what are at least two approaches to acquiring the knowledge or
  mastering the skill?  Of the identified approaches, which will each team
  member pursue, and why did they make this choice?
\end{enumerate}

\end{document}