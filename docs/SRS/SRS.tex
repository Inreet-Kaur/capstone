% THIS DOCUMENT IS TAILORED TO REQUIREMENTS FOR SCIENTIFIC COMPUTING.  IT SHOULDN'T
% BE USED FOR NON-SCIENTIFIC COMPUTING PROJECTS
\documentclass[12pt]{article}

\usepackage{amsmath, mathtools}
\usepackage{amsfonts}
\usepackage{amssymb}
\usepackage{graphicx}
\usepackage{colortbl}
\usepackage{xr}
\usepackage{hyperref}
\usepackage{longtable}
\usepackage{xfrac}
\usepackage{tabularx}
\usepackage{float}
\usepackage{siunitx}
\usepackage{booktabs}
\usepackage{caption}
\usepackage{pdflscape}
\usepackage{afterpage}


\usepackage[round]{natbib}

%\usepackage{refcheck}

\hypersetup{
    bookmarks=true,         % show bookmarks bar?
      colorlinks=true,       % false: boxed links; true: colored links
    linkcolor=red,          % color of internal links (change box color with linkbordercolor)
    citecolor=green,        % color of links to bibliography
    filecolor=magenta,      % color of file links
    urlcolor=cyan           % color of external links
}

\input{../Comments}
%% Common Parts

\newcommand{\progname}{SFWRENG 4G06A} % PUT YOUR PROGRAM NAME HERE
\newcommand{\authname}{Team \#25, RapidCare
\\ Pranav Kalsi
\\ Gurleen Rahi
\\ Inreet Kaur
\\ Moamen Ahmed} % AUTHOR NAMES                  

\usepackage{hyperref}
    \hypersetup{colorlinks=true, linkcolor=blue, citecolor=blue, filecolor=blue,
                urlcolor=blue, unicode=false}
    \urlstyle{same}
                                


% For easy change of table widths
\newcommand{\colZwidth}{1.0\textwidth}
\newcommand{\colAwidth}{0.13\textwidth}
\newcommand{\colBwidth}{0.82\textwidth}
\newcommand{\colCwidth}{0.1\textwidth}
\newcommand{\colDwidth}{0.05\textwidth}
\newcommand{\colEwidth}{0.8\textwidth}
\newcommand{\colFwidth}{0.17\textwidth}
\newcommand{\colGwidth}{0.5\textwidth}
\newcommand{\colHwidth}{0.28\textwidth}

% Used so that cross-references have a meaningful prefix
\newcounter{defnum} %Definition Number
\newcommand{\dthedefnum}{GD\thedefnum}
\newcommand{\dref}[1]{GD\ref{#1}}
\newcounter{datadefnum} %Datadefinition Number
\newcommand{\ddthedatadefnum}{DD\thedatadefnum}
\newcommand{\ddref}[1]{DD\ref{#1}}
\newcounter{theorynum} %Theory Number
\newcommand{\tthetheorynum}{TM\thetheorynum}
\newcommand{\tref}[1]{TM\ref{#1}}
\newcounter{tablenum} %Table Number
\newcommand{\tbthetablenum}{TB\thetablenum}
\newcommand{\tbref}[1]{TB\ref{#1}}
\newcounter{assumpnum} %Assumption Number
\newcommand{\atheassumpnum}{A\theassumpnum}
\newcommand{\aref}[1]{A\ref{#1}}
\newcounter{goalnum} %Goal Number
\newcommand{\gthegoalnum}{GS\thegoalnum}
\newcommand{\gsref}[1]{GS\ref{#1}}
\newcounter{stretchgoalnum} %Stretch Goal Number
\newcommand{\sgthestretchgoalnum}{STG\thestretchgoalnum}
\newcommand{\sgref}[1]{STG\ref{#1}}
\newcounter{instnum} %Instance Number
\newcommand{\itheinstnum}{IM\theinstnum}
\newcommand{\iref}[1]{IM\ref{#1}}
\newcounter{reqnum} %Requirement Number
\newcommand{\rthereqnum}{R\thereqnum}
\newcommand{\rref}[1]{R\ref{#1}}
\newcounter{nfrnum} %NFR Number
\newcommand{\rthenfrnum}{NFR\thenfrnum}
\newcommand{\nfrref}[1]{NFR\ref{#1}}
\newcounter{lcnum} %Likely change number
\newcommand{\lthelcnum}{LC\thelcnum}
\newcommand{\lcref}[1]{LC\ref{#1}}

\usepackage{fullpage}

\newcommand{\deftheory}[9][Not Applicable]
{
\newpage
\noindent \rule{\textwidth}{0.5mm}

\paragraph{RefName: } \textbf{#2} \phantomsection 
\label{#2}

\paragraph{Label:} #3

\noindent \rule{\textwidth}{0.5mm}

\paragraph{Equation:}

#4

\paragraph{Description:}

#5

\paragraph{Notes:}

#6

\paragraph{Source:}

#7

\paragraph{Ref.\ By:}

#8

\paragraph{Preconditions for \hyperref[#2]{#2}:}
\label{#2_precond}

#9

\paragraph{Derivation for \hyperref[#2]{#2}:}
\label{#2_deriv}

#1

\noindent \rule{\textwidth}{0.5mm}

}

\begin{document}

\title{Software Requirements Specification for \progname: RapidCare}
\author{\authname}
\date{\today}
	
\maketitle

~\newpage

\pagenumbering{roman}

\tableofcontents

~\newpage

\section*{Revision History}

\begin{tabularx}{\textwidth}{p{3cm}p{2cm}X}
\toprule {\bf Date} & {\bf Version} & {\bf Notes}\\
\midrule
Date 1 & 1.0 & Notes\\
Date 2 & 1.1 & Notes\\
\bottomrule
\end{tabularx}

~\newpage

\section{Reference Material}

\subsection{Table of Units}
N/A

\subsection{Table of Symbols}
N/A

\subsection{Abbreviations and Acronyms}

\begin{tabular}{l l} 
  \toprule    
  \textbf{symbol} & \textbf{description}\\
  \midrule 
  A & Assumption\\
  G & Goals\\
  STG & Stretch Goals\\
  FR & Functional Requirement\\
  NFR & Non-functional Requirement\\
  LC & Likely Change\\
  ULC & Unlikely Change\\
  SRS & Software Requirements Specification\\
  EHR & Electronic Healthcare Record\\
  \bottomrule
\end{tabular}\\

\subsection{Mathematical Notation}
N/A

\newpage

\pagenumbering{arabic}


\section{Introduction}

\subsection{Purpose of Document} \label{sec_PurposeOfDocument}

The purpose of this document is to provide a comprehensive description of the requirements for a software application that aims to streamline the healthcare documentation process aimed to be run as a web application. This document will be used in as a contract in a sense between the team and the client who intends to use this application. This document will allow for an in-depth description of the software's functionality, performance, and other non-functional requirements. Additionally, it will outline common use-cases under which the software will be used. This will in turn provide a direction to the developers such that they will be empowered to creating the right product as this document will contain various stakeholders' requirements. Along with development direction, this document will be a direct reference for all of the stakeholders to understand the product's scope, functionality, and limitations.

\subsection{Scope of Requirements} \label{sec_ScopeOfRequirements}

The scope of this project can be be split into the following:

\begin{itemize}
  \item \textbf{Out of Scope:}
  \begin{itemize}
    \item The system will not be responsible for patient scheduling, billing, or other administrative tasks.
    \item The system will not include development of any hardware components. All needed peripherals will be assumed as apart of the runtime environment.
  \end{itemize}
  \item \textbf{Typical Values of Input}
  \begin{itemize}
    \item The primary input mode will be voice dictation, allowing for transcription and segmentation of conversations.
    \item The other input is typing through a keyboard to amend or edit any documentation.
    \item Standard office environment is expected, with standard computer equipment (i.e. mice, keyboards, internet).
  \end{itemize}
\end{itemize}


\subsection{Characteristics of Intended Reader} \label{sec_IntendedReader} 

The intended readers of this SRS document include project managers, software developers, testing engineers, and stakeholders directly involved in the design and implementation of the software system. Project managers and testing engineers would be directly involved in the development and testing process. They would generally have an education in computer science along with experience in software design, web or mobile development, and software testing. Project managers generally have experience in managing software projects and knowledge of software development processes. Stakeholders like doctors and nursing staff would have domain knowledge and could provide insights to understand the clinical workflow and documentation process. 

\subsection{Organization of Document} \label{sec_OrganizationOfDocument}

The organization of this document is as follows:
\begin{itemize}
  \item \textbf{System Description}\\
  This section will provide an overview of the system, its functionality, the characteristics of the users, and the constraints of the system.
  \item \textbf{Requirements}\\
  This section will outline the functional and non-functional requirements. Additionally, it will provide a rationale for all of the requirements.
  \item \textbf{Likely Changes}\\
  This section will outline the likely changes that may occur in the future.
  \item \textbf{Unlikely Changes}\\
  This section will outline the unlikely changes that may occur in the future.
  \item \textbf{References}\\
  This section will provide a list of references used in the document.
\end{itemize}

\section{General System Description} \label{sec_GeneralSystemDescription}

\subsection{System Context} \label{sec_SystemContext}

Through this project we aim to develop a solution that will address key niche problems through customizability and add features of critical need that do not already exist in existing solutions. This will allow healthcare networks to centralize the data of their hospitals and increase the staff productivity.\\

\noindent This product has been conceived by the group members based on elicitation. Through interviews and discussion the problem of documentation overhead was found. This is the problem to be solved for the capstone course SFWRENG 4G06.


inputs
Outputs
External Entities
User responsibilities
System responsibilities

\subsubsection{Expected Benefits} \label{sec_ExpectedBenefits}

The expected benefits of this project are as follows:
\begin{itemize}
  \item Reduced documentation overhead time
  \item Increased patient throughput
  \item Improved patient care
  \item Increased doctor and healthcare professional satisfaction
\end{itemize}

Much like how copilot is a tool for programmers, the aim is to create a tool for healthcare professionals to help with documentation such that the number of patient processed per day can be increased, and the focus of doctors and nurses can be shifted from documentation to patient care. 


\subsection{User Characteristics} \label{sec_UserCharacteristics}

The intended users of this product are healthcare professionals, specifically those involved in patient documentation, including doctors, nurses, and other clinical administrative staff. \\

The following are the user characteristics associated with the intended users:\\

\textbf{Education Level:} All users are expected to have a minimum of a college diploma and basic reading, writing, and speaking skills. The education level of the intended users will vary based on their role, generally a bachelor’s degree or college diploma in nursing, medicine, or other healthcare fields. \\

\textbf{Experience:} All users will be expected to have basic familiarity with EHR systems or similar applications. This experience will depend on their roles ranging from entry-level positions to experienced professionals. \\

\textbf{Technical Expertise:} The users should have basic technical skills to use EHR systems and other healthcare software applications. In addition to this, the users should have familiarity with data entry processes and analytics interpretation. \\

\textbf{Accessibility Considerations:} It is anticipated that some users may have accessibility issues. The user interface should be designed for ease of navigation, incorporating various accessibility features.\\

Specific requirements must be specified to accommodate users with varying technical expertise and diverse educational backgrounds. This will ensure the application is usable by all healthcare staff, regardless of their abilities.


\subsection{System Constraints}

The following constraints will guide the system design and implementation:

\begin{itemize} 

  \item \textbf{Compliance with Regulatory Standards:} The system must adhere to healthcare regulations ensuring the confidentiality and security of patient data.

  \item \textbf{User Accessibility:} The system must meet established accessibility standards to ensure usability for all healthcare staff, including those with disabilities.

  \item \textbf{Cost and Time:} The system must be developed within the imposed budget and time restrictions.

\end{itemize}


~\newpage

\section{Specific System Description} \label{sec_SpecificSystemDescription}


\subsection{Problem Description} \label{sec_ProblemDescription}

Ontario is facing an extreme shortage of family doctors, with the number of patients without one jumping by 600,000 to 2.5 million which is a growing number [1]. This situation is only to get worse as predicted by the Ontario Medical Association [2]. As a result, people find themselves going to the ER with coughs and colds and flooding the ER causing massive wait times which ends in patients even resulting in leaving without being seen [3]. A massive part of the wait time is due to the overhead of documentation tasks. Doctors, healthcare professionals, and support staff find themselves spending most of their time on documentation which overall slows the pipeline of patients tremendously.


\subsubsection{Terminology and Definitions} \label{sec_TerminologyDefinitions}

The following is the glossary for this document:

\begin{itemize}
  \item \textbf{System:} When referring to a system the document refers to the intended solution/product.
  \item \textbf{User:} When referring to a user the document refers to a person who will use the product.
  \item \textbf{Healthcare Professional:} When referring to healthcare professionals the document refers to doctors, nurses, and other healthcare professionals who will be using the product.
  \item \textbf{Patient:} When referring to a patient the document refers to any person who is receiving medical treatment.
  \item \textbf{Healthcare Network:} When referring to a healthcare network the document refers to an organisation that has group of hospitals or clinics and will use the system.

    \item[R\refstepcounter{reqnum}\thereqnum \label{FR_DictationRecording}:] \textbf{Voice Dictation for Patient Records}

    \textbf{Requirement:} he system should allow healthcare workers to update patient records using dictation. 
    
    \textbf{Rationale:} This will reduce the workload on the healthcare workers, while also efficiently reducing the time spent on documentation. 
    
    \textbf{Fit Criterion:} The system should use voice dictation to document medical reports with a minimum of 90\% accuracy. 
    
    \textbf{Dependencies:} 3.1.7, 3.1.8
    
    \textbf{Monitored Variables:}
    \begin{itemize}
        \item Accuracy
    \end{itemize}
    
    \textbf{Performance:} 
    \begin{itemize}
        \item System should transcribe speech in real time and segment it into the right fields onto the record.
    \end{itemize}
    
    \textbf{Hardware Requirements:}
    \begin{itemize}
        \item Workstations to access the system and a microphone which can record the voice in the system. 
    \end{itemize}
    
    \textbf{Software Requirements:}
    \begin{itemize}
        \item An audio transcription service to transcribe audio to text 

        \item Access to patient record database. 

        \item Internet browser to access the system.  
    \end{itemize}
    
    \textbf{Normal Use:} The system records the provider's voice and updates the patient file with the transcribed text.
    
    \textbf{Undesired Event Handling:} If the recording or transcription fails, the system should notify the user and offer manual review.



\item[R\refstepcounter{reqnum}\thereqnum \label{FR_DiagnosticSuggestions}:] \textbf{Diagnostic Suggestions from Transcribed Data}

    \textbf{Requirement:} The system should analyze transcribed text and provide diagnostic suggestions based on the input.
    
    \textbf{Rationale:} This alerts the healthcare professionals about the possible diagnostics and helps them to diagnose the patients quicker based on suggested options.
    
    \textbf{Fit Criterion:} The system must suggest diagnoses within 5 seconds after transcript is provided, with 85\% accuracy based on known patient’s medical record.
    
    \textbf{Dependencies:} 3.1.12
    
    \textbf{Monitored Variables:}
    \begin{itemize}
        \item Suggested diagnostic accuracy 
        \item Time to generate suggestions 
    \end{itemize}
    
    \textbf{Performance:}
    \begin{itemize}
        \item Suggestions must be provided within 5 seconds of transcript completion. 
        \item  Suggestions should match known diagnoses with 85\% accuracy. 
    \end{itemize}
    
    \textbf{Hardware Requirements:}
    \begin{itemize}
        \item Workstations and other peripherals to access the system. 
    \end{itemize}
    
    \textbf{Software Requirements:}
    \begin{itemize}
        \item Access to patient record database. 
        \item Internet browser to access the system.  
        \item Service engine that generates diagnostic suggestions. 
    \end{itemize}
    
    \textbf{Normal Use:} The system analyzes the transcribed conversation and suggests possible diagnoses based on a patient’s medical record. 
    
    \textbf{Undesired Event Handling:} If the system can’t suggest a diagnosis, it should notify the user and offer a manual search option. 

\end{itemize}


\subsubsection{Physical System Description} \label{sec_phySystDescrip}
N/A


\subsubsection{Goals Statements} \label{sec_Goals}
The goals for this project are as follows:
\begin{table}[H]
    \centering
    \begin{tabular}{p{4cm} p{4cm} p{4cm}}
        \toprule
        \textbf{Goal} & \textbf{Definition} & \textbf{Rationale} \\
        \midrule
        G\refstepcounter{goalnum}\thegoalnum\label{G_VoiceToDocumentation}: Use voice to fill in medical documentation (charts, files, etc.) & The app will record conversations and automatically turn them into medical notes and charts. & This will save doctors time by automating paperwork, letting them focus more on patients. \\
        \midrule
        G\refstepcounter{goalnum}\thegoalnum \label{G_reduceOverhead}: Reduce documentation overhead time.  & Through tracking the whole patient journey in the app, we look to reduce the overhead of triaging, clinical documentation, and other registrations.  & This helps hospital healthcare professionals focus on care and lowers the time taken through registration for hospital staff. \\ 
        \midrule
        G\refstepcounter{goalnum}\thegoalnum \label{G_integrateEnv}: Integrates with the existing hospital environment. & We want the solution to be portable such that it can be implemented in existing hospital and clinic ecosystems.  & Portability will ensure that hospitals and clinics won’t have to upgrade their existing hardware to use the application. \\
        \midrule 
        G\refstepcounter{goalnum}\thegoalnum \label{G_hNetworkProfiles}: Allow health networks profiles with in service. & Need the ability for health networks to add, update, and delete network profile details. & This will ensure that the list of hospitals and staff in the network is up to date with appropriate permissions. \\
        \midrule 
        G\refstepcounter{goalnum}\thegoalnum \label{G_hProfessionalProfiles}: Allow health care professional profiles with in service. & Need the ability for health networks to add, update, and delete staff profile details. & This will ensure that the list of staff will be able to use the tool to dictate patient conversations or add other notes. Additionally, making sure they have access to patient journey. \\
        \midrule 

    \end{tabular}
\end{table}

\subsubsection{Stretch Goals} \label{sec_StretchGoals}
The stretch goals for this project are as follows:
\begin{table}[H]
    \centering
    \begin{tabular}{p{4cm} p{4cm} p{4cm}}
        \toprule
        \textbf{Goal} & \textbf{Definition} & \textbf{Rationale} \\
        \midrule
        STG\refstepcounter{stretchgoalnum}\thestretchgoalnum \label{STG_medicineSuggestions}: Automated medicine suggestions. & Based on diagnosis and patient data provide medicine suggestions. & This will help doctors fill out their charts faster. \\ % Row 1
        \midrule
        STG\refstepcounter{stretchgoalnum}\thestretchgoalnum \label{STG_diagnosisSuggestions}: Automated diagnosis suggestions.  & Use AI to suggest possible diagnoses based on what the doctor and patient discuss.  & This will help doctors make faster, more accurate diagnoses, especially in tricky cases.\\ 
        \midrule
        STG\refstepcounter{stretchgoalnum}\thestretchgoalnum \label{STG_triage}: Increase efficiency for triage.  & Create functionality to prioritize patients based on the severity of their condition. & This ensures the most critical patients get treated first, improving care in emergencies. \\
        \bottomrule
    \end{tabular}
\end{table}


\subsection{Solution Characteristics Specification} \label{sec_SolutionCharacteristicsSpecification}

This section outlines the core functionalities and characteristics of the proposed system:

\begin{itemize}
  \item \textbf{Patient Documentation:} The system should facilitate efficient and streamlined patient documentation throughout the EHR documentation process. This includes storing patient data, clinical notes, treatment plans, and patient history in a secure manner.
  
  \item \textbf{Voice Transcription:} The system should be able to record conversations between healthcare professionals and patients which will be transcribed to fill out charts and patient profiles efficiently.
  
  \item \textbf{Real-time Data Access:} The users must have real-time access to patient data and documentation enabling them to make informed decisions quickly and efficiently.
  
  \item \textbf{Automated Medicine Suggestions:} The system should be able to provide automated medicine suggestions based on the transcribed data which will help users to fill out the charts faster and make informed treatment decisions.
  
  \item \textbf{Automated Diagnosis Suggestions:} The system should be able to suggest possible diagnoses based on transcribed data which will help users to make faster and more accurate diagnoses.
  
  \item \textbf{User Authentication:} The system must provide secure user authentication methods to ensure that only authorized personnel can access sensitive patient information.
  
\end{itemize}

\subsubsection{Types} \label{sec_Types}
N/A

\subsubsection{Scope Decisions} \label{sec_ScopeDecisions}
N/A

\subsubsection{Modelling Decisions} \label{sec_ModellingDecisions}
N/A


\subsubsection{Assumptions} \label{sec_assumpt}

\begin{itemize}
  \item\textbf{Reliable Internet Connection:} We assume that the user has a reliable internet connection throughout their operational hours.
  \item\textbf{Sufficient Hardware Accessories:} We are assuming that the user has the required hardware devices such as monitors, iPads etc. to access the system.
  \item\textbf{Patient’s Consent:} We also assume that medical staff will obtain patients’ consent when required.
\end{itemize}


\subsubsection{Theoretical Models}\label{sec_theoretical}
N/A

\subsubsection{General Definitions} \label{sec_GeneralDefinitions}
N/A

\subsubsection{Data Definitions}\label{sec_DataDefinitions} 
N/A

\subsubsection{Data Types}\label{sec_DataTypes}
N/A

\subsubsection{Instance Models} \label{sec_InstanceModels} 
N/A

\subsubsection{Input Data Constraints} \label{sec_InputDataConstraints}
N/A


\subsubsection{Properties of a Correct Solution} \label{sec_CorrectSolution}

Along with the functionalities and propertiesmentioned above, here are the use-case scenarios for the system:

\begin{itemize}
  \item\textbf{UC1 Login:}
  \begin{itemize}
    \item The user accesses the system using an internet browser.
    \item The user enters valid credentials on the log in page.
    \item The user selects the login button.
    \item The user lands on the default dashboard.
  \end{itemize}
  \item\textbf{UC2 Recording Clinical Notes:}
  \begin{itemize}
    \item The user accesses patient’s record.
    \item The user selects dictate button.
    \item The user dictates the notes and hit the stop button.
    \item The user reviews the transcribed text.
    \item The user selects the submit button.
  \end{itemize}
  \item\textbf{UC3 Diagnostic Suggestions:}
  \begin{itemize}
    \item The user submits the transcribed text.
    \item The user reviews the potential diagnostic suggestions.
    \item The user accepts or rejects suggestions.
  \end{itemize}
  \item\textbf{UC4 Create Patient Profile:}
  \begin{itemize}
    \item The user log into the system.
    \item The user selects the ‘create a new record’ button.
    \item The user provide input for the required fields.    
    \item The user selects the submit button.
  \end{itemize}
\end{itemize}

\begin{figure}[h]
  \centering
  \includegraphics[width=0.8\textwidth]{use-case.drawio.png}
  \caption{This is the use-case diagram for this scenario.}
  \label{fig:Use-Case Diagram}
\end{figure}

\textbf{Usage Scenario 1: Documenting a Patient Consultation}

\begin{itemize}
  \item\textbf{Use Case:} UC2, UC3
  \item\textbf{Primary Actor:} Healthcare professional (such as a medical doctor or a nurse)
  \item\textbf{Precondition:} The user has been successfully authenticated, logged into the system and has patient’s profile is created. The basic information such as name, contact information, and history is prefilled.
  \item\textbf{Trigger:} The user will press the dictate button.
  \item\textbf{Main Success Scenario:}
  \begin{itemize}
    \item The user will press the dictate button.
    \item The system will convert audio to text in real-time.
    \item The user will press stop button.
    \item The user reviews the transcribed notes. 
    \item The system has accurately transcribed the audio to text without any inaccuracies.
    \item The user selects the submit button after review. 
  \end{itemize}
  \item\textbf{Secondary Success Scenario:}
  \begin{itemize}
    \item The system has produced some inaccuracies in the transcribed text.
    \begin{itemize}
      \item The user selects the edit button.
      \item The system prompts user to edit the text.
      \item The user manually edits the transcribed text.
    \end{itemize} 
  \end{itemize}
  \item\textbf{Success Postcondition:}
  \begin{itemize}
    \item The notes are successfully saved in patient database and changes are reflected in the user interface.
  \end{itemize}
\end{itemize}

\subsubsection{Risks and Mitigation}

Looking at implementation details there are a few risks that come up, if these risks can be addressed or have a clearer roadmap that will make us more confident in the project. 

\begin{itemize}
  \item \textbf{Speech Input:} A hospital or a clinic can be a loud place, in the event audio input is taken we need to ensure that it is clean and clear. This would mean essentially blocking outside noise. 
  \item \textbf{Pre-Trained Models:} To manipulate and use both inputs above we need to create a model to be accurate and provide accuracy when filling in charts. 
  \item \textbf{User Acceptance:} This will require further elicitation from outside supervisors. We need to gather data on what critical needs of healthcare professionals such that critical features are present.
  \item \textbf{Technical Delay:} Integration with EHR systems might cause some delays if there’s any technical issue or if the software faces compatibility issues. This will require sample testing to make sure that the system is compatible in the early stage of the development process.
  \item \textbf{Professional Verification:} Misinterpretation of some words might lead to inaccurate records and wrong diagnosis. Therefore, it’s important for the healthcare professionals to verify the final version of the document and manually delete anything that was falsely recorded. 
\end{itemize}

~\newpage

\section{Requirements} \label{sec_Requirements}


\subsection{Functional Requirements} \label{sec_FunctionalRequirements}


\noindent \begin{itemize}

\item[FR\refstepcounter{reqnum}\thereqnum \label{FR_meaningfulLabel}:] 

\textbf{Requirement:} The system should allow to add healthcare network to the system. 

\textbf{Rationale:} Rapid care is an organizational tool, where networks can register their various hospitals, and in turn add the corresponding health care professionals. When the networks want to register, the app needs to be able to add the network to the database along with the relevant staff profiles etc.

\textbf{Fit Criterion:} The network data and profiles are fully added to the database. This could be verified by returning the valid entries from the patient database.

\textbf{Dependencies:} N/A

\textbf{Monitored and Controlled Variables:} N/A

\textbf{Performance Requirements:} 
\begin{itemize}
  \item The system should update the entries with the latency of 1 second.
  \item After user input is taken the system only adds valid entries to the database and prevents any data leaks.
\end{itemize}

\textbf{Hardware Requirements:} 
\begin{itemize}
  \item Workstations and other peripherals to access the system.
\end{itemize}

\textbf{Software Requirements:} 
\begin{itemize}
  \item Database management system to store health network data.
  \item Internet browser to access the application.
\end{itemize}

\textbf{Normal Behavior:} 
\begin{itemize}
  \item All input data is validated as being entered into the system.
  \item Once all required fields are completed the user selects the submit button and the network is added successfully to the system.
  \item The process should have a low turnover time such that health networks will not have to spend a long time waiting to use the system.
\end{itemize}

\textbf{Undesired Event Handling:} 
\begin{itemize}
  \item The user may enter invalid input data. The system should display appropriate error messages. 
  \item The system should have constraints to restrict the user from submitting, unless all required fields are completed and have valid input data. 
  \item When the database is overloaded with requests, appropriate error messages should be delayed. 
  \item The updates will be queued to prevent this in the future, data resources will be scaled just so that the calls are faster. This will include indexing or caching and scaling the solution horizontally to balance the load.
\end{itemize}


\item[FR\refstepcounter{reqnum}\thereqnum \label{FR_meaningfulLabel}:]  

\textbf{Requirement:} The system should allow to remove a health network from the system.

\textbf{Rationale:} 
When health networks close and want to pivot to another documentation tool their data and profiles must be deleted. Therefore, there is a requirement for functionality that allows organizations to deregister and have their data deleted.

\textbf{Fit Criterion:} 
The network data and profiles are fully deleted from the database. This could be verified by returning the valid entries from the patient database.

\textbf{Dependencies:} 3.1.1

\textbf{Monitored and Controlled Variables:} N/A

\textbf{Performance Requirements:} 
\begin{itemize}
  \item The removal process must be easy to complete with a latency of 1 second. 
  \item The system should be able to identify the correct record to delete. 
  \item The system should delete the correct record without affecting the rest of the database. 
\end{itemize}

\textbf{Hardware Requirements:} 
\begin{itemize}
  \item Workstations and other peripherals to access the system.
\end{itemize}

\textbf{Software Requirements:}
\begin{itemize}
  \item Access to health network database.
  \item Internet browser to access the application.
\end{itemize}

\textbf{Normal Behavior:}
\begin{itemize}
  \item Network is successfully removed from database with low turnover time such that health networks will not have to spend a long time waiting for their data to be deleted.
\end{itemize}

\textbf{Undesired Event Handling:}
\begin{itemize}
  \item If the system fails to delete the health network due to a system error, the system should display an appropriate error message. 
  \item When the database is overloaded with requests, the operation to delete all the hospital data will be queued as the next action in line.
\end{itemize}


\item[FR\refstepcounter{reqnum}\thereqnum \label{FR_UpdateHealthNetwork}:]

\textbf{Requirement:} The health care network should be able to update its organizational and hospital information.

\textbf{Rationale:} The healthcare network will update its own organizational changes. This will include creating and maintaining the staff present as well as the hospitals in the network.

\textbf{Fit Criterion:} The healthcare network data will be up to date with its current state (i.e. number of hospitals, staff etc…), the data’s recency will depend on the networks need. 

\textbf{Dependencies:} 3.1.1 

\textbf{Monitored and Controlled Variables:} N/A

\textbf{Performance Requirements:} 
\begin{itemize}
  \item The updating process of healthcare networks should be quick and easy so that healthcare professionals remain up to date with the facility information, operational data, healthcare professional data, and patient data.
\end{itemize}

\textbf{Hardware Requirements:} 
\begin{itemize}
  \item Workstations and other peripherals to access the system.
\end{itemize}

\textbf{Software Requirements:} 
\begin{itemize}
  \item Internet browser to access the application.
\end{itemize}

\textbf{Normal Behavior:}
\begin{itemize}
  \item Network is updated in the database without any leaks or latency.
  \item Normal behavior will be seen as updated reflected on the front-end and backend of the system.
\end{itemize} 

\textbf{Undesired Event Handling:} 
\begin{itemize}
  \item When the health network data is being updated and the database is overloaded with requests, then updates will be queued to prevent this in the future, data resources will be scaled just that the calls are faster this will include indexing or caching and scaling the solution horizontally to balance the load.
\end{itemize}

\item[FR\refstepcounter{reqnum}\thereqnum \label{FR_AddHealthProfessional}:]

\textbf{Requirement:} The system should allow to add healthcare professionals to the system.

\textbf{Rationale:} When new healthcare professionals join the healthcare network, their information is to be added to the system for authentication purposes. This will include adding a list of hospital staff members to the network.

\textbf{Fit Criterion:} The healthcare professional’s data will be up to date in the system so that they can get authenticated without any delay. 

\textbf{Dependencies:} N/A 

\textbf{Monitored and Controlled Variables:} N/A

\textbf{Performance Requirements:} 
\begin{itemize}
  \item The addition process should be quick and easy to ensure that all professionals can access the database without any interruptions.
\end{itemize}

\textbf{Hardware Requirements:} 
\begin{itemize}
  \item Workstations and other peripherals to access the system.
\end{itemize}

\textbf{Software Requirements:} 
\begin{itemize}
  \item Internet browser to access the database.
\end{itemize}

\textbf{Normal Behavior:}
\begin{itemize}
  \item Data is added to the database without any leaks or latency. Normal behavior will be seen as updated are reflected on the front-end and backend of the system.
\end{itemize} 

\textbf{Undesired Event Handling:}
\begin{itemize}
  \item When the healthcare professional’s data is being added and the database is overloaded with requests, then updates will be queued to prevent this in the future, data resources will be scaled just that the calls are faster this will include indexing or caching and scaling the solution horizontally to balance the load.
\end{itemize} 

\item[FR\refstepcounter{reqnum}\thereqnum \label{FR_RemoveHealthProfessionals}:] 

\textbf{Requirement:} The system should allow to remove healthcare professionals from the system. 

\textbf{Rationale:} Sometimes a healthcare professional decides to change the area of service or leaves the organization and retires. Therefore, there is a need for functionality that allows to remove them from the system.

\textbf{Fit Criterion:} The healthcare professional is successfully removed from the system, this will be verified by the fact that they are not present in the databases. 

\textbf{Dependencies:} 3.1.4 

\textbf{Monitored and Controlled Variables:} N/A

\textbf{Performance Requirements:}
\begin{itemize}
  \item The deleting process must be easy to complete with a low turnover time such that health networks will not have to spend a long time waiting for their data to be deleted.
\end{itemize} 

\textbf{Hardware Requirements:}
\begin{itemize}
  \item Workstations and other peripherals to access the system.
\end{itemize} 

\textbf{Software Requirements:}
\begin{itemize}
  \item Internet browser to access the database.
\end{itemize} 

\textbf{Normal Behavior:}
\begin{itemize}
  \item Data is removed to the database without any leaks or latency. Normal behavior will be seen as updated are reflected on the front-end and backend of the system.
\end{itemize} 

\textbf{Undesired Event Handling:}
\begin{itemize}
  \item When the healthcare professional’s data is being removed and the database is overloaded with requests, then updates will be queued to prevent this in the future, data resources will be scaled just that the calls are faster this will include indexing or caching and scaling the solution horizontally to balance the load.
\end{itemize} 

\item[FR\refstepcounter{reqnum}\thereqnum \label{FR_UpdateHealthProfessionals}:]

\textbf{Requirement:} The system should allow to update healthcare professional’s data in the system.

\textbf{Rationale:} Sometimes healthcare professionals receive promotions or other changes in their roles which require updating the information in their profiles. Therefore, it is essential for functionality that allows to edit the data in case there are inaccuracies.  

\textbf{Fit Criterion:} The healthcare professional’s data will be up to date in its current state (i.e. role of a healthcare professional, the organization they work in etc), the data’s latency will depend on the networks’ need. 

\textbf{Dependencies:} 3.1.4 

\textbf{Monitored and Controlled Variables:} N/A

\textbf{Performance Requirements:} 
\begin{itemize}
  \item The changes should be reflected in real time on the user interface.
  \item The changes should be stored successfully in the database.
\end{itemize} 

\textbf{Hardware Requirements:}
\begin{itemize}
  \item Workstations and other peripherals to access the system.
\end{itemize} 

\textbf{Software Requirements:}
\begin{itemize}
  \item Internet browser to access the database.
\end{itemize} 

\textbf{Normal Behavior:}
\begin{itemize}
  \item Data is updated in the database without any leaks or latency. Normal behavior will be seen as updated are reflected on the front-end and backend of the system.
\end{itemize} 

\textbf{Undesired Event Handling:}
\begin{itemize}
  \item When the healthcare professional’s data is being updated and the database is overloaded with requests, then updates will be queued to prevent this in the future, data resources will be scaled just that the calls are faster this will include indexing or caching and scaling the solution horizontally to balance the load.
\end{itemize} 

\item[FR\refstepcounter{reqnum}\thereqnum \label{FR_login}:]

\textbf{Requirement:} The system should allow the authorized user to successfully log in to the system.

\textbf{Rationale:} Doctors and other medical professionals should be able to log into the system to create patients, update, and delete patient records and access their medical records. The system should also authenticate the user to avoid unauthorized access to the data.

\textbf{Fit Criterion:} The system only authenticates the authorized users to log into the system and has 100\% accuracy. 

\textbf{Dependencies:}  3.1.4

\textbf{Monitored and Controlled Variables:} number of failed login attempts

\textbf{Performance Requirements:} 
\begin{itemize}
  \item The system successfully redirects users to correct system state based on authentication.
\end{itemize}

\textbf{Hardware Requirements:} 
\begin{itemize}
  \item Workstations and other peripherals to access the system.
\end{itemize}

\textbf{Software Requirements:} 
\begin{itemize}
  \item Authentication protocols and encryption for security. 
  \item Internet browser to access the system.
\end{itemize}

\textbf{Normal Behavior:} 
\begin{itemize}
  \item The user is able to successfully login upon providing valid credentials and is redirected to the appropriate dashboard based on their role.
\end{itemize}

\textbf{Undesired Event Handling:}
\begin{itemize}
  \item If a user provides invalid credentials, the system will display an error message and redirect the user to sign in page.
  \item After three failed login attempts, the user account will be locked, and the user will have to contact the support team to regain access.
\end{itemize}
 

\item[FR\refstepcounter{reqnum}\thereqnum \label{FR_createRecord}:]

\textbf{Requirement:} The user should be able to create a new patient record. 

\textbf{Rationale:} The patient data and medical history must be stored in a secure manner. Therefore, medical staff should be able to create a new patient record to store all the relevant information. 

\textbf{Fit Criterion:} The record is created and added to the patient database. This could be verified by returning the valid entries from the patient database.

\textbf{Dependencies:} 3.1.7

\textbf{Monitored and Controlled Variables:} field validation

\textbf{Performance Requirements:} 
\begin{itemize}
  \item The system should update the patient database with the latency of 1 second. 
  \item The system only adds valid entries to the database.
  \item The system prevents any data leaks.
\end{itemize}

\textbf{Hardware Requirements:} 
\begin{itemize}
  \item Workstations and other peripherals to access the system.
\end{itemize}

\textbf{Software Requirements:} 
\begin{itemize}
  \item Database management system to store patient information.
  \item Internet browser to access the system.
\end{itemize}

\textbf{Normal Behavior:} 
\begin{itemize}
  \item All input data is validated as it is entered using field level validation. 
  \item Once all required fields are completed the user selects the submit button, a new patient record is successfully created and stored.
\end{itemize}

\textbf{Undesired Event Handling:} 
\begin{itemize}
  \item The user may enter invalid input data. The system should display appropriate error messages. 
  \item The system should have constraints to restrict the user from submitting, unless all required fields are completed and have valid input data. 
  \item If the system fails to save the record due to a system error, the system should display an appropriate error message. 
\end{itemize}


\item[FR\refstepcounter{reqnum}\thereqnum \label{FR_deleteRecord}:]

\textbf{Requirement:} The user should be able to delete an existing patient record from the system.

\textbf{Rationale:} There can be instances where medical professionals need to delete a certain patient record, for example inaccurate entries, duplicate records etc. Therefore, the system should allow authorized users to remove patient records.

\textbf{Fit Criterion:} The record is successfully removed from the patient database. This could be verified by returning the valid entries from the patient database.

\textbf{Dependencies:} 3.1.7, 3.1.8

\textbf{Monitored and Controlled Variables:} N/A

\textbf{Performance Requirements:} 
\begin{itemize}
  \item The deletion process must be easy to complete with a latency of 1 second. 
  \item The system should be able to identify the correct record to delete. 
  \item The system should delete the correct record without affecting the rest of the database. 
\end{itemize}

\textbf{Hardware Requirements:} 
\begin{itemize}
  \item Workstations and other peripherals to access the system.
\end{itemize}

\textbf{Software Requirements:} 
\begin{itemize}
  \item Access to patient record database.
  \item Role based access control system to manage user permissions.
  \item Internet browser to access the system.
\end{itemize}

\textbf{Normal Behavior:} 
\begin{itemize}
  \item The user selects the delete button and confirms the deletion. The patient record is successfully removed from the database and no longer appears on the system.
\end{itemize}

\textbf{Undesired Event Handling:} 
\begin{itemize}
  \item If the user does not have permission to delete the record, the system should show an appropriate error message.
  \item If the system fails to delete the record due to a system error, the system should display an appropriate error message. 
\end{itemize}



\item[FR\refstepcounter{reqnum}\thereqnum \label{FR_updateRecordtyping}:]

\textbf{Requirement:} : The system should allow the user to update the patient records manually by typing.

\textbf{Rationale:} Medical professionals frequently need to update patient information, such as changes in diagnosis, medication, or medical history. The healthcare professional may want to add some information manually. Moreover, if healthcare professionals use the dictation tool, they still should be allowed to edit the auto filled transcribed data in case there are inaccuracies.

\textbf{Fit Criterion:} The system should update the current state and patient database with the input information.

\textbf{Dependencies:} 3.1.7, 3.1.8

\textbf{Monitored and Controlled Variables:} N/A

\textbf{Performance Requirements:} 
\begin{itemize}
  \item The changes should be reflected in real time on the user interface.
  \item The changes should be stored successfully in the database.
\end{itemize}

\textbf{Hardware Requirements:} 
\begin{itemize}
  \item Workstations and other peripherals to access the system.
\end{itemize}

\textbf{Software Requirements:} 
\begin{itemize}
  \item Access to patient record database.
  \item Internet browser to access the system. 
\end{itemize}

\textbf{Normal Behavior:} 
\begin{itemize}
  \item The user edits the selected field and enters new information. 
  \item The system successfully updates the changes in database and reflect changes on the user interface. 
\end{itemize}

\textbf{Undesired Event Handling:}
\begin{itemize}
  \item If an unauthorized user tries to update a record, the system displays appropriate error messages. 
  \item If the system fails to update the record due to a system error, the system should display an appropriate error message. 
\end{itemize}


\item[FR\refstepcounter{reqnum}\thereqnum \label{FR_meaningfulLabel}:] 

\textbf{Requirement:} Update the patient record by dictation -ability to record. 

\textbf{Fit Criterion:}  

\textbf{Dependencies:}  

\textbf{Monitored and Controlled Variables:} 

\textbf{Performance Requirements:} 

\textbf{Hardware Requirements:} 

\textbf{Software Requirements:} 

\textbf{Normal Behavior:} 

\textbf{Undesired Event Handling:} 

\item[FR\refstepcounter{reqnum}\thereqnum \label{FR_meaningfulLabel}:] 

\textbf{Requirement:} transcribe conversation to text in real time. 

\textbf{Fit Criterion:}  

\textbf{Dependencies:}  

\textbf{Monitored and Controlled Variables:} 

\textbf{Performance Requirements:} 

\textbf{Hardware Requirements:} 

\textbf{Software Requirements:} 

\textbf{Normal Behavior:} 

\textbf{Undesired Event Handling:} 

\item[FR\refstepcounter{reqnum}\thereqnum \label{FR_meaningfulLabel}:] 

\textbf{Requirement:} provide diagnostic suggestions based on the transcribed data.

\textbf{Fit Criterion:}  

\textbf{Dependencies:}  

\textbf{Monitored and Controlled Variables:} 

\textbf{Performance Requirements:} 

\textbf{Hardware Requirements:} 

\textbf{Software Requirements:} 

\textbf{Normal Behavior:} 

\textbf{Undesired Event Handling:} 

\item[FR\refstepcounter{reqnum}\thereqnum \label{FR_meaningfulLabel}:] 
\textbf{Requirement:} The system should provide a list of frequently used medicines based on accepted diagnosis

\textbf{Rationale:} As the patients’ medical records and data is being added into the charts, having a model that can provide a preliminary set of diagnosis such that documentation time can be saved. This will be a supplementary auto-complete type of tool again to continue to speed up documentation.

\textbf{Fit Criterion:} Auto-completion accuracy is 85+ percent as that the benchmark for most autocompleting keyboards etc. The model will be measured through cross validation techniques.

\textbf{ Dependencies:} 3.1.7, 3.1.8, 3.1.10, 3.1.11, 3.1.12

\textbf{Monitored and Controlled Variables:} accuracy

\textbf{Performance Requirements:}
\begin{itemize}
  \item The autocompletion is accurate, such that it is correct 85+ percent of the time. 
\end{itemize}

\textbf{Hardware Requirements:} 
Workstations and other peripherals to access the system.

\textbf{Software Requirements:}
\begin{itemize}
  \item Access to patient record database.
  \item Internet browser to access the system. 
\end{itemize}

\textbf{Normal Behavior:}
\begin{itemize}
  \item A list of medicine suggestion is provided almost instantly analogous to an auto-complete feature based on the selected diagnosis.
\end{itemize}

\textbf{Undesired Event Handling:}
\begin{itemize}
  \item In case no medicine suggestions are provided the doctor can manually add a prescription.
\end{itemize}

\item[FR\refstepcounter{reqnum}\thereqnum \label{FR_meaningfulLabel}:] 
\textbf{Requirement:} The system should be able to transfer patient documentation to another provider. 

\textbf{Rationale:} If the patient goes to a hospital in another health network, perhaps the medical history is to be shared. This functionality will allow the request and transfer of files.

\textbf{Fit Criterion:} Transfer over the files in a short time, the transfer protocol is secure, and data is not leaked.

\textbf{Dependencies:} 3.1.1, 3.1.4, 3.1.8, 3.1.10, 3.1.11, 3.1.12

\textbf{Monitored and Controlled Variables:} N/A

\textbf{Performance Requirements:}
\begin{itemize}
  \item The system should transfer data with low latency.
\end{itemize}

\textbf{Hardware Requirements:} 
\begin{itemize}
  \item Workstations and other peripherals to access the system.
\end{itemize}

\textbf{Software Requirements:}
Access to patient record database.
\begin{itemize}
  \item Internet browser to access the system. 
\end{itemize}

\textbf{Normal Behavior:}
\begin{itemize}
  \item A patient file is requested, the request is approved then the files are transferred.
\end{itemize}

\textbf{Undesired Event Handling:}
\begin{itemize}
  \item A patient file is requested, the request is denied then a detailed request is submitted.
\end{itemize}

\item[FR\refstepcounter{reqnum}\thereqnum \label{FR_meaningfulLabel}:] 
\textbf{Requirement:} provide a list of frequently used medicines based on diagnosis.

\textbf{Fit Criterion:}  

\textbf{Dependencies:}  

\textbf{Monitored and Controlled Variables:} 

\textbf{Performance Requirements:} 

\textbf{Hardware Requirements:} 

\textbf{Software Requirements:} 

\textbf{Normal Behavior:} 

\textbf{Undesired Event Handling:} 

\item[FR\refstepcounter{reqnum}\thereqnum \label{FR_meaningfulLabel}:] 

\textbf{Requirement:} send patient journey to another provider.

\textbf{Fit Criterion:}  

\textbf{Dependencies:}  

\textbf{Monitored and Controlled Variables:} 

\textbf{Performance Requirements:} 

\textbf{Hardware Requirements:} 

\textbf{Software Requirements:} 

\textbf{Normal Behavior:} 

\textbf{Undesired Event Handling:} 

\end{itemize}


\subsection{Non-functional Requirements} \label{sec_NonFunctionalRequirements}

\noindent \begin{itemize}

\item[NFR\refstepcounter{nfrnum}\thenfrnum \label{NFR_LookAndFeel}:] \textbf{Aesthetic and Design}

    \textbf{Requirement:} The UI should keep a clean design, that fits the healthcare standards.

    \textbf{Rationale:} A clean user interface allows users to navigate through the application with ease.

    \textbf{Fit Criterion:} UI demos will be sampled to healthcare workers to ensure that the design is easily understood and user-friendly.

    \textbf{Dependencies:} Design feedback loops and UX/UI design software.  

    \textbf{Undesired Event Handling:} If the surveys shows less than 80\% satisfaction, the design will then be revised.


\item[NFR\refstepcounter{nfrnum}\thenfrnum \label{NFR_Usability}:] \textbf{Usability}

    \textbf{Requirement:} The UI of the system should be intuitive, allowing healthcare workers to master its use with a training of up to 30 minutes. 

    \textbf{Rationale:} If the UI is user-friendly, it tends to reduce the learning time and allows health care workers to focus on patients.
  

    \textbf{Fit Criterion:} Tests that demonstrates that atleast 90\% of healthcare workers are able to navigate the system with 30 minutes of training or less.

    \textbf{Dependencies:} Feedback on the design during the development of the system.

    \textbf{Undesired Event Handling:} The system should support the user using a live chat support integrated within the application.

\item[NFR\refstepcounter{nfrnum}\thenfrnum \label{NFR_Performance}:] \textbf{Performance}

    \textbf{Requirement:} The system should convert voice recordings into text onto medical charts within 30 seconds of recording.

    \textbf{Rationale:} Reduces the documentation time to reduce workload and allow healthcare workers to focus on patients.
    
    \textbf{Fit Criterion:} The system will consistently generate completed documentation within 30 seconds of recording completion.  

    \textbf{Dependencies:} Speech-to-text engine.  

    \textbf{Undesired Event Handling:} The system will reduce any background processes, if it fails to generate completed documentation within 30 seconds.

\item[NFR\refstepcounter{nfrnum}\thenfrnum \label{NFR_Operational}:] \textbf{Operational Requirement}

    \textbf{Requirement:} The system should have an uptime guarantee of 99.9\% during operational hours.

    \textbf{Rationale:} Reliable uptime ensures consistent system for healthcare workers, specially in time-sensitive atmospheres like hospitals.

    \textbf{Fit Criterion:} System logs will confirm an uptime rate of 99.9\% over a 30-day period.  

    \textbf{Dependencies:} Cloud infrastructure and local hardware resilience. 
    
    \textbf{Undesired Event Handling:} There should be backup servers that turn on within 10 seconds of downtime to ensure reliable uptime.

\item[NFR\refstepcounter{nfrnum}\thenfrnum \label{NFR_Maintainability}:] \textbf{Maintainability Requirement}

    \textbf{Requirement:} The system should be updated with new features without causing downtime longer than 2 minutes.

    \textbf{Rationale:} Easily maintained system allows for consistent service.

    \textbf{Fit Criterion:} Release logs will show updates occur with less than 2 minutes of downtime.  

    \textbf{Dependencies:} Continuous integration of updates to the system.  

    \textbf{Undesired Event Handling:} If an update fails, the system will automatically revert to the last stable version.

\item[NFR\refstepcounter{nfrnum}\thenfrnum \label{NFR_Security}:] \textbf{Security Requirement}

    \textbf{Requirement:} The system should have all patient data encrypted, and in compliance with HIPAA standards.

    \textbf{Rationale:} A secure and confidential system ensures the user to be confident in using the application.

    \textbf{Fit Criterion:} Security audits will show 100\% compliance with HIPAA and encryption standards.  

    \textbf{Dependencies:} Encryption services and security protocols.  

    \textbf{Undesired Event Handling:} If a security breach is detected, all users will be logged out, access will be locked, and administrators alerted.

\item[NFR\refstepcounter{nfrnum}\thenfrnum \label{NFR_Cultural}:] \textbf{Cultural Requirement}

    \textbf{Requirement:} The system should allow customization of language settings to accommodate all healthcare workers.

    \textbf{Rationale:} A customized UI allows the user to feel comfortable and confident when using the application.

    \textbf{Fit Criterion:} Tests will show that healthcare workers can navigate through the application and switch languages based on their preference without impacting the functionality of the application.

    \textbf{Dependencies:} Language bundles.  

    \textbf{Undesired Event Handling:} If a language bundle fails to load, the system will revert to the default language and notify the user.

\item[NFR\refstepcounter{nfrnum}\thenfrnum \label{NFR_Legal}:] \textbf{Legal Requirement}

    \textbf{Requirement:} The system should comply with all healthcare data protection regulations. 

    \textbf{Rationale:} Legal compliance is mandatory to protect patient data and avoid penalties.  

    \textbf{Fit Criterion:} Legal audits will confirm compliance with HIPAA and any other relevant data protection laws.  

    \textbf{Dependencies:} Legal consultancy, data protection regulations.  

    \textbf{Undesired Event Handling:} If the system fails to comply with regulations, it will be updated within 24 hours to meet compliance.

\end{itemize}

\subsection{Rationale}

\plt{Provide a rationale for the decisions made in the documentation.  Rationale
should be provided for scope decisions, modelling decisions, assumptions and
typical values.}

\section{Likely Changes}    

\noindent \begin{itemize}

\item[LC\refstepcounter{lcnum}\thelcnum\label{LC_meaningfulLabel}:] \plt{Give
    the likely changes, with a reference to the related assumption (aref), as appropriate.}

\end{itemize}

\section{Unlikely Changes}    

\noindent \begin{itemize}

\item[ULC\refstepcounter{lcnum}\thelcnum\label{LC_meaningfulLabel}:] \plt{Give
    the unlikely changes.  The design can assume that the changes listed will
    not occur.}

\end{itemize}


\section{Traceability Matrices and Graphs}
N/A   

\section{Development Plan}
N/A

\section{Values of Auxiliary Constants}
N/A


~\newpage

\section{References}

\newpage{}
\section*{Appendix --- Reflection}

\wss{Not required for CAS 741}

The information in this section will be used to evaluate the team members on the
graduate attribute of Lifelong Learning.  

\input{../Reflection.tex}

\begin{enumerate}
  \item What went well while writing this deliverable? 
  \item What pain points did you experience during this deliverable, and how did
  you resolve them?
  \item How many of your requirements were inspired by speaking to your
  client(s) or their proxies (e.g. your peers, stakeholders, potential users)?
  \item Which of the courses you have taken, or are currently taking, will help
  your team to be successful with your capstone project.
  \item What knowledge and skills will the team collectively need to acquire to
  successfully complete this capstone project?  Examples of possible knowledge
  to acquire include domain specific knowledge from the domain of your
  application, or software engineering knowledge, mechatronics knowledge or
  computer science knowledge.  Skills may be related to technology, or writing,
  or presentation, or team management, etc.  You should look to identify at
  least one item for each team member.
  \item For each of the knowledge areas and skills identified in the previous
  question, what are at least two approaches to acquiring the knowledge or
  mastering the skill?  Of the identified approaches, which will each team
  member pursue, and why did they make this choice?
\end{enumerate}

\end{document}