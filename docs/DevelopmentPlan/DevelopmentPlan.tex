\documentclass{article}

\usepackage{booktabs}
\usepackage{tabularx}
\usepackage{hyperref}

\title{Development Plan\\\progname}

\author{\authname}

\date{}

%% Comments

\usepackage{color}

\newif\ifcomments\commentstrue %displays comments
%\newif\ifcomments\commentsfalse %so that comments do not display

\ifcomments
\newcommand{\authornote}[3]{\textcolor{#1}{[#3 ---#2]}}
\newcommand{\todo}[1]{\textcolor{red}{[TODO: #1]}}
\else
\newcommand{\authornote}[3]{}
\newcommand{\todo}[1]{}
\fi

\newcommand{\wss}[1]{\authornote{blue}{SS}{#1}} 
\newcommand{\plt}[1]{\authornote{magenta}{TPLT}{#1}} %For explanation of the template
\newcommand{\an}[1]{\authornote{cyan}{Author}{#1}}

%% Common Parts

\newcommand{\progname}{ProgName} % PUT YOUR PROGRAM NAME HERE
\newcommand{\authname}{Team \#, Team Name
\\ Student 1 name
\\ Student 2 name
\\ Student 3 name
\\ Student 4 name} % AUTHOR NAMES                  

\usepackage{hyperref}
    \hypersetup{colorlinks=true, linkcolor=blue, citecolor=blue, filecolor=blue,
                urlcolor=blue, unicode=false}
    \urlstyle{same}
                                


\begin{document}


\maketitle

\begin{table}[hp]
\caption{Revision History} \label{TblRevisionHistory}
\begin{tabularx}{\textwidth}{llX}
\toprule
\textbf{Date} & \textbf{Developer(s)} & \textbf{Change}\\
\midrule
23--09--2024 & Inreet & Introduction\\
23--09--2024 & Inreet & Confidential Information\\
23--09--2024 & Inreet & IP to Protect\\
23--09--2024 & Gurleen & Copyright License\\
23--09--2024 & Gurleen & Team meeting Plan\\
23--09--2024 & Gurleen & Team Communication Plan\\
23--09--2024 & Inreet & Team member Roles\\
23--09--2024 & Inreet, Pranav & Workflow Plan\\
23--09--2024 & Inreet & Project Decomposition and Scheduling\\
23--09--2024 & Pranav & Proof of Concept Demonstration Plan\\
23--09--2024 & Pranav & Expected Technology\\
23--09--2024 & Gurleen & Coding Standards\\
23--09--2024 & All & Reflection\\
23--09--2024 & Inreet & Team Charter\\
23--09--2024 & Pranav & Review + Fix Compilation Errors\\
... & ... & ...\\
\bottomrule
\end{tabularx}
\end{table}

\newpage{}

This document will provide details about various essential aspects of our development plan. It covers information about critical areas such as confidential information, IPs, and copyright considerations ensuring compliance with ethical standards and regulatory requirements. In addition to this, we will outline the team meeting plan and team communication plan, and team member roles. We will include a workflow plan that describes how we plan to use Git and GitHub projects and project decompositions and scheduling. Moreover, we have included a proof-of-concept demonstration plan along with the technology stack and coding standards we wish to use for the development of our project. Finally, we will reflect on our learning from this exercise and include a team charter that defines our goals, expectations, and decision-making strategy.

\section{Confidential Information?}

Our project would not use any confidential data and information from the industry. We plan to use publicly available data in the development of this application, ensuring there is no breach of confidentiality.

\section{IP to Protect}

There is no IP to protect this project at the current stage. All work during the development phase will be original and no existing patents or proprietary technologies are involved. 

\section{Copyright License}

\wss{What copyright license is your team adopting.  Point to the license in your
repo.}

The team would like to choose a Proprietary License for this project. This means that the copyright holder retains all rights and cannot use, copy and modify the code. The license file in our Github repository has been updated with the same.

\section{Team Meeting Plan}

\wss{How often will you meet? where?}

\wss{If the meeting is a physical location (not virtual), out of an abundance of
caution for safety reasons you shouldn't put the location online}

\wss{How often will you meet with your industry advisor?  when?  where?}

\wss{Will meetings be virtual?  At least some meetings should likely be
in-person.}

\wss{How will the meetings be structured?  There should be a chair for all meetings.  There should be an agenda for all meetings.}

The team will have a virtual meeting (on Microsoft teams) to discuss different stages throughout the duration of the project. We will meet at least once every week to discuss the state of the project as well as update ourselves on other team members’ work. Meetings with the industry advisor will depend on their availability, however, the mode of meeting will be in-person. We create an issue for team meetings in the repository on GitHub where we list the agenda of the meeting, meeting minutes and action items, i.e., the next steps.

\section{Team Communication Plan}

\wss{Issues on GitHub should be part of your communication plan.}

The team members will communicate with each other using Microsoft Teams for team meetings. Before every team meeting, an issue is created in the github repository which contains the attendance, meeting agenda, meeting minutes and action items. The issue is closed once the meeting is done. 
For submissions according to the team roles, we have made our branches and submitting our respective work in those branches. At the end, those branches are merged with the main branch and all the latex documents are compiled. 

\section{Team Member Roles}
 
The team will assign specific roles for the project after the preliminary elicitation and design thinking process is complete. Currently, the team would have the following distribution of administrative roles.

\begin{itemize}
\item \textbf{Project Leader} -- Pranav Kalsi
\begin{itemize}
\item Prepare agenda for team meetings, ensuring all relevant topics are covered and goals are clear
\item Assign issues on the Kanban board, tracking the progress of each task and redistributing work if needed
\item Act as a liaison between the group and instructional team
\end{itemize}

\item \textbf{Notetaker} -- Gurleen Rahi
\begin{itemize}
\item Record meeting minutes and attendance during team meetings
\item Assist in updating the Kanban board based on meeting outcomes
\end{itemize}

\item \textbf{Communication Laison} -- Moamen Ahmed
\begin{itemize}
\item Organize team meetings and send reminders to all team members
\item Schedule meetings with external stakeholders
\end{itemize}

\item \textbf{Administrator} -- Inreet Kaur
\begin{itemize}
\item Submit Deliverables on the avenue and ensure they meet the requirements
\item Maintain project documentation and files
\end{itemize}
\end{itemize} 

The team will remain flexible for the following roles, with responsibilities subject to change based on the needs of each deliverable and the final project:
\begin{itemize}
\item \textbf{Meeting Chair} -- This role will rotate between team members. The chair will organize and lead meetings, ensuring that the agenda is followed, and discussions stay focused.
\item \textbf{Reviewer} -- The team plans to distribute each milestone evenly among the team members. At least two other team members will review each team member’s work. The reviewer will update issues on GitHub with appropriate comments.
\item \textbf{Subject researcher} -- During a deliverable team members may be assigned to gather information about the industry, their processes, and information about the technology used in development. This role will shift based on individual expertise.
\end{itemize}
By adopting a flexible approach to the following roles, we want to ensure that everyone has the opportunity to lead. Regular check-ins during the team meetings will allow the team to reassess the roles and redistribute tasks as needed. 


\section{Workflow Plan}

We will utilize Git as our version control throughout this project. Here is a detailed breakdown of how we collaborate using different features on GitHub. 

\begin{itemize}
\item \textbf{Branches} -- We will create different branches related to each task or issue. 
\begin{itemize}
\item Main branch -- The main branch will have a stable production-ready code. 
\item Feature branches -- All features will be assigned their own branch. Once tested it will be merged into the main branch.
\item Documentation branches -- Each team member will create their branch when working on a particular documentation task. 
\end{itemize}

\item \textbf{Pull requests} -- Once a feature is ready for production or some documentation is ready, team members will create a pull request to merge them into the main branch. Each commit should include a brief description of the changes and related issue numbers of the Kanban board.

\item \textbf{Issue tracking} -- The team will utilize the Kanban board under projects for issue tracking. The team will use the provided templates for the appropriate issues and create its own where needed. 
\end{itemize}

The team will be using the following projects:

\begin{itemize}
\item \textbf{Metrics} -- This project will be used to record metrics such as attended lectures and team meetings. It has two types of issues, lecture meetings and team meetings and uses the appropriate templates provided. Each team member adds their own attendance and updates team meeting notes as per their assigned roles.
\item \textbf{Documentation} -- This project is used to create issues for each assigned task related to project documentation. Each issue will be reviewed by at least two other team members. Each issue uses a customized documentation template and is classified based on different milestones of project documentation. 
\item \textbf{Development workflow} -- This project will be used to record issues for different features, feature enhancements, bug fixes, and other needs of the project. The issues will be divided into the above listed categories and will use customized templates for each. 
\end{itemize}

CI/CD will be critical for project development. We will be using \textbf{Jenkins} which is an open-source CI/CD development tool that will act as an automated DevOps tool.

Jenkins will allow us to achieve Continuous Integration and Continuous Delivery through the following:
\begin{itemize}
  \item \textbf{Continuous Integration} -- On the continuous integration time Jenkins offers automated build and testing. This will save a lot of overhead on testing and test feedback as Jenkins will be responsible for it not only will it build and test the code it will also offer instant feedback to the developer.
  \item \textbf{Continuous Deployment} -- Jenkins offers automated deployment functionalities meaning that once the build passes all tests it can be deployed into a production or pre-production environment. This will ensure the deployment is consistent, reliable, and efficient. This also will make sure that all features pass a minimum functionality standard ensuring that they are ready for production. Jenkins offers integrations with git as well so deployments will be version-controlled meaning they may be reverted as needed.
\end{itemize}

By implementing a CI/CD tool (Jenkins) we can ensure that code isn't riddled with errors and automate a lot of tasks which will increase productivity. Having Jenkins also will reduce the risk of human error in the project and will automate many areas of DevOps.

\section{Project Decomposition and Scheduling}

The team will utilize Kanban Board Projects for issue tracking with four different stages of an issue: ‘To Do’, ‘In Progress’, ‘Review’, and ‘Done’. The team will create different issues for features, bugs, documentation, team meetings, lecture meetings and other things as needed using appropriate templates. The team will create templates for features, bugs and documentation. Issues under different projects will be classified into different milestones with appropriate deadlines. This will help us to track our progress for major milestones in the project. Furthermore, we will divide the Kanban board issues into sprints. Therefore, we can reduce the risk of the overall project. Sprints will ensure that the risk of the project is lowered as we will be working in an agile fashion.

Project Links:
\begin{itemize}
\item  
\href{https://github.com/users/Inreet-Kaur/projects/4}{Metrics}
\item  
\href{https://github.com/users/Inreet-Kaur/projects/2}{Documentation} 
\item 
\href{https://github.com/users/Inreet-Kaur/projects/6}{Development workflow}
\end{itemize}

Here is a general process that our team will follow to work on each issue. The team leader will create different issues under the ‘To do’ section and assign them to different team members as discussed during team meetings. Each team member creates their branch and will pull the latest changes from the main branch. Once the team member has made their changes they can be pushed to GitHub and the related issue can be moved under the ‘Review’ tab. The reviewers will provide appropriate comments and suggested changes and move the issue to the ‘In progress’ section if needed. At least two other team members will review each issue. Once the final changes are approved, a PR request will be raised to merge the changes into the main branch and the issue will now be moved to the ‘Done’ section. 

Each task within the three projects will be scheduled according to the deadlines outlined in the course outline. The documentation project will have a separate milestone for each deliverable mentioned in the course outline along with the appropriate deadlines. The metrics project will track all the team meetings, lecture attendance, and other metrics required throughout the course. We will create an issue for each lecture and each team meeting with the required details in the template. The development project will be classified into different milestones throughout this course. Some stages include initial feature planning, Phase 1 which includes developing core functionalities, including voice recording and chart filling, and Phase 2 which includes developing additional features like the analytical dashboard, triage integration etc. Each phase will have separate issues for feature development, bug fixes, enhancement, and final reviews.  

\section{Proof of Concept Demonstration Plan}

% What is the main risk, or risks, for the success of your project?  What will you
% demonstrate during your proof of concept demonstration to convince yourself that
% you will be able to overcome this risk?\\

The main risk of the project is related to the reduction of documentation overhead. This means the biggest risk is really if this project can make a significant reduction to overhead time and truly reduce patient wait times. Again, through elicitation and our domain experts, we will create an effective set of requirements to reduce the risk of the project and build something that is needed.\\

Through our development plan, the coding standards, and the agile approach the development risk is made to be lower than a herculean approach. Therefore due to our development approach and using an agile workflow the project's risk will be reduced as features will be created over time.\\

Looking at implementation details there are a few risks that come up, if these risks can be addressed or have a clearer roadmap that will make us more confident in the project. 

\begin{itemize}
  \item \textbf{Speech Input} -- A hospital or a clinic can be a loud place, in the event audio input is taken we need to ensure that it is clean and clear. This would mean essentially blocking outside noise. 
  \item \textbf{Visual Inputs} -- If any old charts need to be inputted into the patient journey having the ability to scan and transfer the information into the required format will be another risk. The document needs to be rid of noisy data.
  \item \textbf{Pre-Trained Models} -- To manipulate and use both inputs above we need to create a model to be accurate and provide accuracy when filling in charts. 
  \item \textbf{Data privacy} -- This application will hold a lot of patient data so creating a store that is secure and making sure standard data security practice is applied is a must.
  \item \textbf{User Acceptance} -- This will require further elicitation outside supervisors. We need to gather data on what critical needs of healthcare professionals such that critical features are present. 
\end{itemize}

\section{Expected Technology}
% \wss{What programming language or languages do you expect to use?  What external
% libraries?  What frameworks?  What technologies.  Are there major components of
% the implementation that you expect you will implement, despite the existence of
% libraries that provide the required functionality.  For projects with machine
% learning, will you use pre-trained models, or be training your own model?  }
% \wss{The implementation decisions can, and likely will, change over the course
% of the project.  The initial documentation should be written in an abstract way;
% it should be agnostic of the implementation choices, unless the implementation
% choices are project constraints.  However, recording our initial thoughts on
% implementation helps understand the challenge level and feasibility of a
% project.  It may also help with early identification of areas where project
% members will need to augment their training.}
% Topics to discuss include the following:

In terms of expected technologies, we need to expect will again change will implementation details and architectures. Below is a starting point for a microservices-based architecture.

\begin{itemize}
  \item Programming languages/Frameworks and Data management:
    \begin{itemize}
      \item ReactJS -- This will be used for the front end of the application.
      \item Python (Flask) -- This will be used to write backend services.
      \item Java (SpringBoot) -- This will also provide an alternative option for backend services. It could be specifically useful for MongoDB integration.
      \item MongoDB -- All data will be managed through a MongoDB database as it offers built-in caching and performs well under load.
      \item Distributed Cache (Redis) -- Items that make frequent calls can be cached through Redis. This will overall increase the performance of the application.
    \end{itemize}
    

  \item Libraries
    Key frameworks:
    \begin{itemize}
      \item LangChain -- This will be used for LLM integrations as it offers integrations will all the big LLMs.
      \item NLTK -- NLTK will be used for any natural language processing needs. 
      \item Redux -- Redux will be critical in managing frontend states.
      \item OAuth -- OAuth will be critical in microservices communication to validate the services. This will especially come in handy in a specific middleware layer is implemented.
    \end{itemize}

  \item Specific linter tool -- Linters are useful for catching errors early in the development process. This also ensures the code that is written adheres to the standard which will mitigate code smells and enforce a specific style. Linters we can use include PyLint (Python), PMD for error catching, and/or Checkstyle to ensure the required coding standard is upheld.
  \item Specific unit testing framework -- For frontend testing, we can use selenium to interact with the front and leverage frameworks like JUnit for backend unit testing. 
    
  \item Investigation of code coverage measuring tools -- language-specific code coverage tools will be used such as Coverage.py for Python to assess coverage metrics and which cases are tested for.
  \item Continuous integration -- As stated above Jenkins will be used to automate a lot of CI tasks. Please refer to the CI/CD section above.

  \item Tools we will be using include Git, GitHub, and GitHub projects for development. This will mean using version control as be the workflow plan above, and planning features and sprints based on GitHub projects. 
\end{itemize}


\wss{git, GitHub and GitHub projects should be part of your technology.}

\section{Coding Standard}

\wss{What coding standard will you adopt?}

Our team has decided to move forward with PEP 8 coding standard for Python. For Python, PEP 8 has emerged as the style guide that most projects adhere to; it promotes a very readable and eye-pleasing coding style [1]. It also helps to format the code properly and get rid of inconsistent spaces. By using this coding style, we will ensure that our code is clean and understandable and consistent throughout the project.

\begin {itemize}
\item
\href{https://www.geeksforgeeks.org/pep-8-coding-style-guide-python/}
\end {itemize}

\newpage{}

\section*{Appendix --- Reflection}

\wss{Not required for CAS 741}

The purpose of reflection questions is to give you a chance to assess your own
learning and that of your group as a whole, and to find ways to improve in the
future. Reflection is an important part of the learning process.  Reflection is
also an essential component of a successful software development process.  

Reflections are most interesting and useful when they're honest, even if the
stories they tell are imperfect. You will be marked based on your depth of
thought and analysis, and not based on the content of the reflections
themselves. Thus, for full marks we encourage you to answer openly and honestly
and to avoid simply writing ``what you think the evaluator wants to hear.''

Please answer the following questions.  Some questions can be answered on the
team level, but where appropriate, each team member should write their own
response:


\begin{enumerate}
    \item Why is it important to create a development plan prior to starting the
    project?
    \item In your opinion, what are the advantages and disadvantages of using
    CI/CD?
    \item What disagreements did your group have in this deliverable, if any,
    and how did you resolve them?
\end{enumerate}

\newpage{}

\section*{Appendix --- Team Charter}

\wss{borrows from
\href{https://engineering.up.edu/industry_partnerships/files/team-charter.pdf}
{University of Portland Team Charter}}

\subsection*{External Goals}

\wss{What are your team's external goals for this project? These are not the
goals related to the functionality or quality fo the project.  These are the
goals on what the team wishes to achieve with the project.  Potential goals are
to win a prize at the Capstone EXPO, or to have something to talk about in
interviews, or to get an A+, etc.}

\subsection*{Attendance}

\subsubsection*{Expectations}

\wss{What are your team's expectations regarding meeting attendance (being on
time, leaving early, missing meetings, etc.)?}

\subsubsection*{Acceptable Excuse}

\wss{What constitutes an acceptable excuse for missing a meeting or a deadline?
What types of excuses will not be considered acceptable?}

\subsubsection*{In Case of Emergency}

\wss{What process will team members follow if they have an emergency and cannot
attend a team meeting or complete their individual work promised for a team
deliverable?}

\subsection*{Accountability and Teamwork}

\subsubsection*{Quality} 

\wss{What are your team's expectations regarding the quality
of team members' preparation for team meetings and the quality of the
deliverables that members bring to the team?}

\subsubsection*{Attitude}

\wss{What are your team's expectations regarding team members' ideas,
interactions with the team, cooperation, attitudes, and anything else regarding
team member contributions?  Do you want to introduce a code of conduct?  Do you
want a conflict resolution plan?  Can adopt existing codes of conduct.}

\subsubsection*{Stay on Track}

\wss{What methods will be used to keep the team on track? How will your team
ensure that members contribute as expected to the team and that the team
performs as expected? How will your team reward members who do well and manage
members whose performance is below expectations?  What are the consequences for
someone not contributing their fair share?}

\wss{You may wish to use the project management metrics collected for the TA and
instructor for this.}

\wss{You can set target metrics for attendance, commits, etc.  What are the
consequences if someone doesn't hit their targets?  Do they need to bring the
coffee to the next team meeting?  Does the team need to make an appointment with
their TA, or the instructor?  Are there incentives for reaching targets early?}

\subsubsection*{Team Building}

\wss{How will you build team cohesion (fun time, group rituals, etc.)? }

\subsubsection*{Decision Making} 

\wss{How will you make decisions in your group? Consensus?  Vote? How will you
handle disagreements? }

\end{document}