\documentclass{article}

\usepackage{booktabs}
\usepackage{tabularx}
\usepackage{hyperref}

\usepackage{amsmath, mathtools}
\usepackage{amsfonts}
\usepackage{amssymb}
\usepackage{graphicx}
\usepackage{colortbl}
\usepackage{xr}
\usepackage{longtable}
\usepackage{xfrac}
\usepackage{float}
\usepackage{siunitx}
\usepackage{caption}
\usepackage{pdflscape}
\usepackage{afterpage}
\usepackage{float}
\usepackage[round]{natbib}

% For easy change of table widths
\newcommand{\colZwidth}{1.0\textwidth}
\newcommand{\colAwidth}{0.13\textwidth}
\newcommand{\colBwidth}{0.82\textwidth}
\newcommand{\colCwidth}{0.1\textwidth}
\newcommand{\colDwidth}{0.05\textwidth}
\newcommand{\colEwidth}{0.8\textwidth}
\newcommand{\colFwidth}{0.17\textwidth}
\newcommand{\colGwidth}{0.5\textwidth}
\newcommand{\colHwidth}{0.28\textwidth}

\newcounter{srnum} %SR Number
\newcommand{\rthesrnum}{SR\thesrnum}
\newcommand{\srref}[1]{SR\ref{#1}}

\hypersetup{
    colorlinks=true,       % false: boxed links; true: colored links
    linkcolor=red,          % color of internal links (change box color with linkbordercolor)
    citecolor=green,        % color of links to bibliography
    filecolor=magenta,      % color of file links
    urlcolor=cyan           % color of external links
}

\title{Hazard Analysis\\\progname}

\author{\authname}

\date{}

%% Comments

\usepackage{color}

\newif\ifcomments\commentstrue %displays comments
%\newif\ifcomments\commentsfalse %so that comments do not display

\ifcomments
\newcommand{\authornote}[3]{\textcolor{#1}{[#3 ---#2]}}
\newcommand{\todo}[1]{\textcolor{red}{[TODO: #1]}}
\else
\newcommand{\authornote}[3]{}
\newcommand{\todo}[1]{}
\fi

\newcommand{\wss}[1]{\authornote{blue}{SS}{#1}} 
\newcommand{\plt}[1]{\authornote{magenta}{TPLT}{#1}} %For explanation of the template
\newcommand{\an}[1]{\authornote{cyan}{Author}{#1}}

%% Common Parts

\newcommand{\progname}{ProgName} % PUT YOUR PROGRAM NAME HERE
\newcommand{\authname}{Team \#, Team Name
\\ Student 1 name
\\ Student 2 name
\\ Student 3 name
\\ Student 4 name} % AUTHOR NAMES                  

\usepackage{hyperref}
    \hypersetup{colorlinks=true, linkcolor=blue, citecolor=blue, filecolor=blue,
                urlcolor=blue, unicode=false}
    \urlstyle{same}
                                


\begin{document}

\maketitle
\thispagestyle{empty}

~\newpage

\pagenumbering{roman}

\begin{table}[hp]
\caption{Revision History} \label{TblRevisionHistory}
\begin{tabularx}{\textwidth}{llX}
\toprule
\textbf{Date} & \textbf{Developer(s)} & \textbf{Change}\\
\midrule
Date1 & Name(s) & Description of changes\\
Date2 & Name(s) & Description of changes\\
... & ... & ...\\
\bottomrule
\end{tabularx}
\end{table}

~\newpage

\tableofcontents

~\newpage

\pagenumbering{arabic}

\wss{You are free to modify this template.}

\section{Introduction}

\wss{You can include your definition of what a hazard is here.}

\section{Scope and Purpose of Hazard Analysis}

\wss{You should say what \textbf{loss} could be incurred because of the
hazards.}

\section{System Boundaries and Components}

To identify potential hazards, we first define the system boundaries and break it down into its major components:

\begin{itemize}
    \item \textbf{User Interface:}
    The user interface is the point of interaction between the users and the system. It is responsible for displaying outputs from the system, such as patient data, medication suggestions, diagnosis predictions etc. The UI plays a crucial role in ensuring a user-friendly and intuitive experience for the users.
    

    Potential Hazards:
    \begin{itemize}
        \item User errors: incorrect data input or misinterpretation of displayed data
        \item Inadequate feedback when errors occur
        \item Incorrect data displayed to the user
    \end{itemize}

    \item \textbf{Data Layer:}
    The data layer in the system is responsible for managing and processing all data related to patient records, healthcare professionals, health networks, and predictive models for medication and diagnosis. It is divided into the following databases:
    \begin{itemize}
        \item DB1: Patient, Healthcare Professional, and Network Database: This database stores patient records, healthcare professionals, and healthcare network profiles. This component is responsible for storing, retrieving, updating, and deleting data. 

        \item DB2: Diagnosis Prediction Database: This database stores the data used by the diagnosis prediction component to suggest potential diagnoses based on analysis of the transcribed data.

        \item DB3: Medication Prediction Database: This database holds the data used by the medical prediction component to suggest appropriate medications based on the identified or accepted diagnosis.
    \end{itemize}

    Potential Hazards:
    \begin{itemize}
        \item Accidental deletion of database entries or the entire database
        \item Creation of duplicate records
        \item Security breaches
        \item Database crashes   
    \end{itemize}

    \item \textbf{Report Generating Module:}
    The report generating module is responsible for generating organized and thorough reports from the audio conversations recorded during patient's visit. The important aspects of medical data (such as the symptoms, illness history, etc.) is extracted from the conversation which is then compiled into the report using this module.

    Potential Hazards:
    \begin{itemize}
        \item Data inaccuracy: Misinterpretation of a medical term from the audio conversation
        \item Formatting issues
        \item System crash while generating the report
    \end{itemize}

    \item \textbf{Transcription Module:}
    The transcription module is responsible for converting audio data from the conversation to written text. The converted written text is used thereafter used by the report generation module to generate the report of the patient.

    Potential Hazards:
    \begin{itemize}
        \item Background noise disruption
        \item Failure to use correct punctuation for non-verbal signs such as a sigh or pause
        \item Autocorrect failure after rephrasing the word in audio input
    \end{itemize}
    
\end{itemize}


\section{Critical Assumptions}


The following assumptions are made regarding both the software and hardware components of the system:

\begin{itemize}
    \item \textbf{Stable Network Connection:} It is assumed that the network connection between the client and server will be stable. If the connection is unstable, it could cause interruptions to the process, which results in significant issues in the system’s performance.
    
    \item \textbf{Accurate Data Entry:} It is expected that healthcare professionals will enter accurate information into the system. Any incorrect data could lead to inaccurate predictions of patient conditions, potentially creating serious risks.
    
    \item \textbf{Reliable Hardware:} It is assumed that there won’t be any major hardware failures. Although hardware problems are rare, they could severely affect system availability and accuracy, especially in critical healthcare environment.
    
\end{itemize}


\begin{landscape} 

    \section{Failure Mode and Effect Analysis}
    
    \begin{longtable}{|p{1.5cm}|p{2cm}|p{2.6cm}|p{2cm}|p{2cm}|p{2cm}|p{3.5cm}|p{1cm}|p{0.8cm}|}
        \toprule
        \textbf{Comp.} & \textbf{Design Function} & \textbf{Failure mode} & \textbf{Effects of failure} & \textbf{Causes of failure} & \textbf{Detection} & \textbf{Recommended action} & \textbf{Req.} & \textbf{Ref.}\\ 
        \midrule
        User Interface & \raggedright Allow user input and access data & \raggedright User errors in data input & \raggedright Incorrect data stored in the database; Inaccurate data may lead to medical errors & \raggedright Poor UI design; Lack of input validation & \raggedright User reports; Record validation checks & \raggedright Display soft feedback to guide user input. Implement input masks, field-level validation, and page-level validation to prevent the system from saving any invalid data. Implement constraints on input data fields. & NFR1; NFR2; SR1 & H1.1 \\ 
        \midrule
        & & \raggedright Misinterpretation of displayed data & \raggedright Misdiagnosis; Other medical errors & \raggedright Poor UI design & \raggedright Feedback mechanisms; Detected medical errors & \raggedright Improve UI design for clarity. Improve discoverability and use appropriate signifiers for various data fields. & NFR1; NFR2; SR1 & H1.2 \\ 
        \midrule
        & \raggedright Display error messages and provide feedback & \raggedright Inadequate feedback when errors occur & \raggedright Users are unaware of the current system state; Unresolved issues; Inaccurate data stored in a database & \raggedright Insufficient feedback mechanism & \raggedright Error logs; User reports; Record validation checks & \raggedright Provide clear and actionable error messages when an error occurs. Use language familiar to the user for easy interpretation. Provide steps to recover from the error state & NFR1; NFR2; SR1 & H1.3 \\
        \midrule
        & \raggedright Display correct data to the user & \raggedright Incorrect data displayed to the user & \raggedright Incorrect medical decisions; Compromise patient safety & \raggedright Data processing error; System bugs & \raggedright User reports; Error logs & \raggedright Ensure user input is accurately interpreted and stored by the system. Add data verification steps to ensure the system retrieves the correct data to display & SR2 & H1.4 \\
        \midrule
        Data Layer & \raggedright Manage and store data in a secure manner & \raggedright Accidental deletion of database entries or the entire database & \raggedright Permanent loss of critical data & \raggedright User error; Lack of validation checks & \raggedright User reports; Failure to retrieve or access a data instance or database & \raggedright Display appropriate feedback before confirming the deletion. Implement role-based access control for deletion action. Implement automatic data backup and recovery system. & FR5; FR9; FR2 & H2.1 \\ 
        \midrule
        & & \raggedright Creation of duplicate records & \raggedright Incorrect output displayed to the user; Medical errors & \raggedright Lack of validation on user input & \raggedright Record validation checks & \raggedright Implement validation checks for user input. Implement validation checks before storing a new entry. Regular data integrity checks & SR2 & H2.2 \\ 
        \midrule
        & & \raggedright Security breaches & \raggedright Unauthorized access to sensitive data; Regulatory and compliance issues & \raggedright Improper authentication and encryption & \raggedright Security audits; Access logs & \raggedright Implement strong authentication protocols. Encrypt sensitive data using standard encryption protocols. Ensure compliance with HIPAA and regulatory standards. & NFR6; NFR8; FR7 & H2.3 \\ 
        \midrule
        & \raggedright Retrieve and store data in real-time. & \raggedright Database crashes & \raggedright Inability to access stored data; Inability to store new data & \raggedright Server overload; System failure & \raggedright Error messages; Monitoring system performance & \raggedright Implement failover systems. Implement automatic backups. Implement scalable server infrastructure. & NFR4; NFR5 & H2.4 \\ 
        \midrule
        General & \raggedright Provide continuous access to the system & \raggedright App closes unexpectedly & \raggedright Unsaved progress is lost; Delayed medical access to patients & \raggedright Loss of power or internet; Device failure; Software failure & \raggedright User reports; System logs & \raggedright Implement automatic data backups and recovery system. & NFR4 & H3.1 \\
        \midrule
        API Module (OAuth) & \raggedright Securely connect components with OAuth & \raggedright Failed connection between components & \raggedright Inability to authenticate users, disrupting services & \raggedright Network failure, OAuth misconfiguration & \raggedright Monitor connection status and failed authentication attempts & \raggedright Check network status, verify OAuth configuration, retry connection & ACR1 & H5.1 \\
        \midrule
        User Authentication & \raggedright Verify user credentials & \raggedright User cannot log in to the system & \raggedright User cannot access any system data or functions & \raggedright Invalid credentials, database failure & \raggedright Failed login attempts trigger security alerts & \raggedright Reset credentials, verify database connectivity & IR1 & H6.1 \\
        \midrule
        Account Management & \raggedright Manage user accounts (create, update, delete) & \raggedright Account cannot be created, updated, or deleted & \raggedright User unable to register, update info, or remove account & \raggedright Database failure, validation errors & \raggedright Log account creation, update, and deletion attempts & \raggedright Check database integrity, validate inputs, retry operations & ACR2 & H7.1 \\ 
        \midrule
        Report Generating Module & \raggedright Generate organized medical reports from audio input & \raggedright System crash while report generation & \raggedright Might lead to data loss or delay in fetching information again for compiling & \raggedright Loss of internet access & \raggedright Compiling verification checks & \raggedright The system should check if the compiling can be done successfully while the written notes are being made. & SR4 \\
        \midrule
        Transcription Module & \raggedright Convert audio data from the conversation to written text & \raggedright Failure to use correct punctuation for non-verbal signs such as a sigh or pause & \raggedright Inclusion of incorrect data in the report & \raggedright Inability to filter the background noise as a result of which the conversation is overwritten by the noise & \raggedright Background noise Filteration & \raggedright Filter any noise which is not related to the medical terminologies used in the report and capture every aspect of conversation. This includes using a full stop after a long pause and a comma after a short pause. & SR3 \\ 
        \bottomrule
    \end{longtable}
\end{landscape}

\section{Safety and Security Requirements}


\begin{itemize}
    \item[SR\refstepcounter{srnum}\thesrnum \label{SR_ErrorDetection}:] The system should provide real-time error detection based on validation checks and provide feedback to users.
    \textbf{Rationale:}To prevent user errors and incorrect output it is vital to check errors induced by the user input. Moreover, in the event of an error, the system should communicate its current state, how the input has been interpreted, and any related errors to the user.
    
    \item [SR\refstepcounter{srnum}\thesrnum \label{SR_DuplicateRecordDetection}:] The system should provide duplicate record detection and prevention for patients, healthcare professionals, and healthcare network records.
    \textbf{Rationale:} To prevent confusion and medical errors resulting from duplicate entries, the system should validate and flag potential duplicate records before they are created.

    \item [SR\refstepcounter{srnum}\thesrnum \label{SR_BackNoiseFilter}:] The system should be able to filter the background noise to capture the medical data from the audio conversation between the patient and healthcare professional only.
    \textbf{Rationale:} To prevent data inaccuracy with addition of unwanted information in the report.

    \item [SR\refstepcounter{srnum}\thesrnum \label{SR_CompilingVerification}:] The system should provide validation that the text can be compiled by the report generating module as audio is being transcribed to the written text.
    \textbf{Rationale:} To identify technical errors in the beginning itself. However, in the event of any error, the system should be able to fix it without causing delay in the transcription process. 
\end{itemize}

\subsection{Access Requirements}
\textbf{ACR1}: The API Module (OAuth) must allow only authenticated users access to system resources. Failed authentication attempts must be logged.

\noindent \textbf{ACR2}: Only authorized personnel can create, update, or delete user accounts. Unauthorized actions should be blocked and logged.

\subsection{Integrity Requirements}
\textbf{IR1}: User credentials must remain intact during authentication. Failed login attempts should not affect the system's functionality or stored data.



\section{Roadmap}

\wss{Which safety requirements will be implemented as part of the capstone timeline?
Which requirements will be implemented in the future?}

\newpage{}

\section*{Appendix --- Reflection}

\wss{Not required for CAS 741}

The purpose of reflection questions is to give you a chance to assess your own
learning and that of your group as a whole, and to find ways to improve in the
future. Reflection is an important part of the learning process.  Reflection is
also an essential component of a successful software development process.  

Reflections are most interesting and useful when they're honest, even if the
stories they tell are imperfect. You will be marked based on your depth of
thought and analysis, and not based on the content of the reflections
themselves. Thus, for full marks we encourage you to answer openly and honestly
and to avoid simply writing ``what you think the evaluator wants to hear.''

Please answer the following questions.  Some questions can be answered on the
team level, but where appropriate, each team member should write their own
response:


\begin{enumerate}
    \item What went well while writing this deliverable? 
    \item What pain points did you experience during this deliverable, and how
    did you resolve them?
    \item Which of your listed risks had your team thought of before this
    deliverable, and which did you think of while doing this deliverable? For
    the latter ones (ones you thought of while doing the Hazard Analysis), how
    did they come about?
    \item Other than the risk of physical harm (some projects may not have any
    appreciable risks of this form), list at least 2 other types of risk in
    software products. Why are they important to consider?
\end{enumerate}

\end{document}