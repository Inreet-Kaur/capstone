\documentclass{article}

\usepackage{booktabs}
\usepackage{tabularx}
\usepackage{hyperref}

\hypersetup{
    colorlinks=true,       % false: boxed links; true: colored links
    linkcolor=red,          % color of internal links (change box color with linkbordercolor)
    citecolor=green,        % color of links to bibliography
    filecolor=magenta,      % color of file links
    urlcolor=cyan           % color of external links
}

\title{Hazard Analysis\\\progname}

\author{\authname}

\date{}

\input{../Comments}
%% Common Parts

\newcommand{\progname}{SFWRENG 4G06A} % PUT YOUR PROGRAM NAME HERE
\newcommand{\authname}{Team \#25, RapidCare
\\ Pranav Kalsi
\\ Gurleen Rahi
\\ Inreet Kaur
\\ Moamen Ahmed} % AUTHOR NAMES                  

\usepackage{hyperref}
    \hypersetup{colorlinks=true, linkcolor=blue, citecolor=blue, filecolor=blue,
                urlcolor=blue, unicode=false}
    \urlstyle{same}
                                


\begin{document}

\maketitle
\thispagestyle{empty}

~\newpage

\pagenumbering{roman}

\begin{table}[hp]
\caption{Revision History} \label{TblRevisionHistory}
\begin{tabularx}{\textwidth}{llX}
\toprule
\textbf{Date} & \textbf{Developer(s)} & \textbf{Change}\\
\midrule
Date1 & Name(s) & Description of changes\\
Date2 & Name(s) & Description of changes\\
... & ... & ...\\
\bottomrule
\end{tabularx}
\end{table}

~\newpage

\tableofcontents

~\newpage

\pagenumbering{arabic}

\wss{You are free to modify this template.}

\section{Introduction}

\wss{You can include your definition of what a hazard is here.}

\section{Scope and Purpose of Hazard Analysis}

\wss{You should say what \textbf{loss} could be incurred because of the
hazards.}

\section{System Boundaries and Components}

To identify potential hazards, we first define the system boundaries and break it down into its major components:

\begin{itemize}
    \item \textbf{User Interface:}
    The user interface is the point of interaction between the users and the system. It is responsible for displaying outputs from the system, such as patient data, medication suggestions, diagnosis predictions etc. The UI plays a crucial role in ensuring a user-friendly and intuitive experience for the users.
    

    Potential Hazards:
    \begin{itemize}
        \item User errors: incorrect data input or misinterpretation of displayed data
        \item Inadequate feedback when errors occur
        \item Incorrect data displayed to the user
    \end{itemize}

    \item \textbf{Data Layer:}
    The data layer in the system is responsible for managing and processing all data related to patient records, healthcare professionals, health networks, and predictive models for medication and diagnosis. It is divided into the following databases:
    \begin{itemize}
        \item DB1: Patient, Healthcare Professional, and Network Database: This database stores patient records, healthcare professionals, and healthcare network profiles. This component is responsible for storing, retrieving, updating, and deleting data. 

        \item DB2: Diagnosis Prediction Database: This database stores the data used by the diagnosis prediction component to suggest potential diagnoses based on analysis of the transcribed data.

        \item DB3: Medication Prediction Database: This database holds the data used by the medical prediction component to suggest appropriate medications based on the identified or accepted diagnosis.
    \end{itemize}

    Potential Hazards:
    \begin{itemize}
        \item Accidental deletion of database entries or the entire database
        \item Creation of duplicate records
        \item Security breaches
        \item Database crashes   
    \end{itemize}
    
\end{itemize}


\section{Critical Assumptions}

\wss{These assumptions that are made about the software or system.  You should
minimize the number of assumptions that remove potential hazards.  For instance,
you could assume a part will never fail, but it is generally better to include
this potential failure mode.}

\section{Failure Mode and Effect Analysis}

\wss{Include your FMEA table here. This is the most important part of this document.}
\wss{The safety requirements in the table do not have to have the prefix SR.
The most important thing is to show traceability to your SRS. You might trace to
requirements you have already written, or you might need to add new
requirements.}
\wss{If no safety requirement can be devised, other mitigation strategies can be
entered in the table, including strategies involving providing additional
documentation, and/or test cases.}

\section{Safety and Security Requirements}

\wss{Newly discovered requirements.  These should also be added to the SRS.  (A
rationale design process how and why to fake it.)}

\section{Roadmap}

\wss{Which safety requirements will be implemented as part of the capstone timeline?
Which requirements will be implemented in the future?}

\newpage{}

\section*{Appendix --- Reflection}

\wss{Not required for CAS 741}

\input{../Reflection.tex}

\begin{enumerate}
    \item What went well while writing this deliverable? 
    \item What pain points did you experience during this deliverable, and how
    did you resolve them?
    \item Which of your listed risks had your team thought of before this
    deliverable, and which did you think of while doing this deliverable? For
    the latter ones (ones you thought of while doing the Hazard Analysis), how
    did they come about?
    \item Other than the risk of physical harm (some projects may not have any
    appreciable risks of this form), list at least 2 other types of risk in
    software products. Why are they important to consider?
\end{enumerate}

\end{document}