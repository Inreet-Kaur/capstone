\documentclass[12pt, titlepage]{article}

\usepackage{booktabs}
\usepackage{longtable}
\usepackage{pdflscape}
\usepackage{tabularx}
\usepackage{hyperref}
\hypersetup{
    colorlinks,
    citecolor=blue,
    filecolor=black,
    linkcolor=red,
    urlcolor=blue
}
\usepackage[round]{natbib}

%% Comments

\usepackage{color}

\newif\ifcomments\commentstrue %displays comments
%\newif\ifcomments\commentsfalse %so that comments do not display

\ifcomments
\newcommand{\authornote}[3]{\textcolor{#1}{[#3 ---#2]}}
\newcommand{\todo}[1]{\textcolor{red}{[TODO: #1]}}
\else
\newcommand{\authornote}[3]{}
\newcommand{\todo}[1]{}
\fi

\newcommand{\wss}[1]{\authornote{blue}{SS}{#1}} 
\newcommand{\plt}[1]{\authornote{magenta}{TPLT}{#1}} %For explanation of the template
\newcommand{\an}[1]{\authornote{cyan}{Author}{#1}}

%% Common Parts

\newcommand{\progname}{ProgName} % PUT YOUR PROGRAM NAME HERE
\newcommand{\authname}{Team \#, Team Name
\\ Student 1 name
\\ Student 2 name
\\ Student 3 name
\\ Student 4 name} % AUTHOR NAMES                  

\usepackage{hyperref}
    \hypersetup{colorlinks=true, linkcolor=blue, citecolor=blue, filecolor=blue,
                urlcolor=blue, unicode=false}
    \urlstyle{same}
                                


\begin{document}

\title{System Verification and Validation Plan for \progname{}} 
\author{\authname}
\date{\today}
	
\maketitle

\pagenumbering{roman}

\section*{Revision History}

\begin{tabularx}{\textwidth}{p{3cm}p{2cm}X}
\toprule {\bf Date} & {\bf Version} & {\bf Notes}\\
\midrule
Date 1 & 1.0 & Notes\\
Date 2 & 1.1 & Notes\\
\bottomrule
\end{tabularx}

~\\
\wss{The intention of the VnV plan is to increase confidence in the software.
However, this does not mean listing every verification and validation technique
that has ever been devised.  The VnV plan should also be a \textbf{feasible}
plan. Execution of the plan should be possible with the time and team available.
If the full plan cannot be completed during the time available, it can either be
modified to ``fake it'', or a better solution is to add a section describing
what work has been completed and what work is still planned for the future.}

\wss{The VnV plan is typically started after the requirements stage, but before
the design stage.  This means that the sections related to unit testing cannot
initially be completed.  The sections will be filled in after the design stage
is complete.  the final version of the VnV plan should have all sections filled
in.}

\newpage

\tableofcontents

\listoftables
\wss{Remove this section if it isn't needed}

\listoffigures
\wss{Remove this section if it isn't needed}

\newpage

\section{Symbols, Abbreviations, and Acronyms}

\renewcommand{\arraystretch}{1.2}
\begin{tabular}{l l} 
  \toprule		
  \textbf{symbol} & \textbf{description}\\
  \midrule 
  T & Test\\
  \bottomrule
\end{tabular}\\

\wss{symbols, abbreviations, or acronyms --- you can simply reference the SRS
  \citep{SRS} tables, if appropriate}

\wss{Remove this section if it isn't needed}

\newpage

\pagenumbering{arabic}

This document ... \wss{provide an introductory blurb and roadmap of the
  Verification and Validation plan}

\section{General Information}

\subsection{Summary}

\wss{Say what software is being tested.  Give its name and a brief overview of
  its general functions.}

\subsection{Objectives}

\wss{State what is intended to be accomplished.  The objective will be around
  the qualities that are most important for your project.  You might have
  something like: ``build confidence in the software correctness,''
  ``demonstrate adequate usability.'' etc.  You won't list all of the qualities,
  just those that are most important.}

\wss{You should also list the objectives that are out of scope.  You don't have 
the resources to do everything, so what will you be leaving out.  For instance, 
if you are not going to verify the quality of usability, state this.  It is also 
worthwhile to justify why the objectives are left out.}

\wss{The objectives are important because they highlight that you are aware of 
limitations in your resources for verification and validation.  You can't do everything, 
so what are you going to prioritize?  As an example, if your system depends on an 
external library, you can explicitly state that you will assume that external library 
has already been verified by its implementation team.}

\subsection{Challenge Level and Extras}

\wss{State the challenge level (advanced, general, basic) for your project.
Your challenge level should exactly match what is included in your problem
statement.  This should be the challenge level agreed on between you and the
course instructor.  You can use a pull request to update your challenge level
(in TeamComposition.csv or Repos.csv) if your plan changes as a result of the
VnV planning exercise.}

\wss{Summarize the extras (if any) that were tackled by this project.  Extras
can include usability testing, code walkthroughs, user documentation, formal
proof, GenderMag personas, Design Thinking, etc.  Extras should have already
been approved by the course instructor as included in your problem statement.
You can use a pull request to update your extras (in TeamComposition.csv or
Repos.csv) if your plan changes as a result of the VnV planning exercise.}

\subsection{Relevant Documentation}

\wss{Reference relevant documentation.  This will definitely include your SRS
  and your other project documents (design documents, like MG, MIS, etc).  You
  can include these even before they are written, since by the time the project
  is done, they will be written.  You can create BibTeX entries for your
  documents and within those entries include a hyperlink to the documents.}

\citet{SRS}

\wss{Don't just list the other documents.  You should explain why they are relevant and 
how they relate to your VnV efforts.}

\section{Plan}

\wss{Introduce this section.  You can provide a roadmap of the sections to
  come.}

\subsection{Verification and Validation Team}

\wss{Your teammates.  Maybe your supervisor.
  You should do more than list names.  You should say what each person's role is
  for the project's verification.  A table is a good way to summarize this information.}

\subsection{SRS Verification Plan}

\wss{List any approaches you intend to use for SRS verification.  This may
  include ad hoc feedback from reviewers, like your classmates (like your
  primary reviewer), or you may plan for something more rigorous/systematic.}

\wss{If you have a supervisor for the project, you shouldn't just say they will
read over the SRS.  You should explain your structured approach to the review.
Will you have a meeting?  What will you present?  What questions will you ask?
Will you give them instructions for a task-based inspection?  Will you use your
issue tracker?}

\wss{Maybe create an SRS checklist?}

\subsection{Design Verification Plan}

\wss{Plans for design verification}

\wss{The review will include reviews by your classmates}

\wss{Create a checklists?}

\subsection{Verification and Validation Plan Verification Plan}

\wss{The verification and validation plan is an artifact that should also be
verified.  Techniques for this include review and mutation testing.}

\wss{The review will include reviews by your classmates}

\wss{Create a checklists?}

\subsection{Implementation Verification Plan}

\wss{You should at least point to the tests listed in this document and the unit
  testing plan.}

\wss{In this section you would also give any details of any plans for static
  verification of the implementation.  Potential techniques include code
  walkthroughs, code inspection, static analyzers, etc.}

\wss{The final class presentation in CAS 741 could be used as a code
walkthrough.  There is also a possibility of using the final presentation (in
CAS741) for a partial usability survey.}

\subsection{Automated Testing and Verification Tools}

\wss{What tools are you using for automated testing.  Likely a unit testing
  framework and maybe a profiling tool, like ValGrind.  Other possible tools
  include a static analyzer, make, continuous integration tools, test coverage
  tools, etc.  Explain your plans for summarizing code coverage metrics.
  Linters are another important class of tools.  For the programming language
  you select, you should look at the available linters.  There may also be tools
  that verify that coding standards have been respected, like flake9 for
  Python.}

\wss{If you have already done this in the development plan, you can point to
that document.}

\wss{The details of this section will likely evolve as you get closer to the
  implementation.}

\subsection{Software Validation Plan}

\wss{If there is any external data that can be used for validation, you should
  point to it here.  If there are no plans for validation, you should state that
  here.}

\wss{You might want to use review sessions with the stakeholder to check that
the requirements document captures the right requirements.  Maybe task based
inspection?}

\wss{For those capstone teams with an external supervisor, the Rev 0 demo should 
be used as an opportunity to validate the requirements.  You should plan on 
demonstrating your project to your supervisor shortly after the scheduled Rev 0 demo.  
The feedback from your supervisor will be very useful for improving your project.}

\wss{For teams without an external supervisor, user testing can serve the same purpose 
as a Rev 0 demo for the supervisor.}

\wss{This section might reference back to the SRS verification section.}

\section{System Tests}

\wss{There should be text between all headings, even if it is just a roadmap of
the contents of the subsections.}

\subsection{Tests for Functional Requirements}

\wss{Subsets of the tests may be in related, so this section is divided into
  different areas.  If there are no identifiable subsets for the tests, this
  level of document structure can be removed.}

\wss{Include a blurb here to explain why the subsections below
  cover the requirements.  References to the SRS would be good here.}

\subsubsection{Area of Testing1}

\wss{It would be nice to have a blurb here to explain why the subsections below
  cover the requirements.  References to the SRS would be good here.  If a section
  covers tests for input constraints, you should reference the data constraints
  table in the SRS.}
		
\paragraph{Title for Test}

\begin{enumerate}

\item{test-id1\\}

Control: Manual versus Automatic
					
Initial State: 
					
Input: 
					
Output: \wss{The expected result for the given inputs.  Output is not how you
are going to return the results of the test.  The output is the expected
result.}

Test Case Derivation: \wss{Justify the expected value given in the Output field}
					
How test will be performed: 
					
\item{test-id2\\}

Control: Manual versus Automatic
					
Initial State: 
					
Input: 
					
Output: \wss{The expected result for the given inputs}

Test Case Derivation: \wss{Justify the expected value given in the Output field}

How test will be performed: 

\end{enumerate}

\subsubsection{Area of Testing2}

...

\subsection{Tests for Nonfunctional Requirements}

\subsubsection{Aesthetic and Design (NFR1)}
		
\paragraph{Aesthetic and Design}

\begin{enumerate}
    \item{AD1\\}
    
    Type: Non-functional, Dynamic, Manual\\
    
    Initial State: UI is designed and implemented.\\
    
    Input/Condition: Healthcare workers view the UI under normal operating conditions.\\
    
    Output/Result: Feedback collected on UI’s aesthetic appeal and simplicity.\\
    
    How this test will be performed: A test group of users and the supervisor will be given the system, and a set of routine UI interactions to perform. They will be asked to complete these interactions using the interface. After they do this, they will be given a UI design survey. In the UI design survey, the supervisor will verify that the interface meets aesthetic appeal and simplicity requirements.


    \item{AD2\\}
    
    Type: Non-functional, Dynamic, Manual\\
    
    Initial State: UI is operational and accessible to users.\\
    
    Input/Condition: Users interact with the UI during routine tasks and provides feedback.\\
    
    Output/Result: 80\% of users report satisfaction with the UI design.\\
    
    How this test will be performed: A test group of users and the supervisor will be given the system, and a set of common tasks to perform. They will be asked to complete these tasks using the interface. After they do this, they will be given a UI satisfaction survey. In the UI satisfaction survey, the supervisor will verify that at least 80\% of users are satisfied with the design.
\end{enumerate}

\subsubsection{Usability Requirement(NFR2)}

\paragraph{Usability}

\begin{enumerate}
    \item{UR1\\}
    
    Type: Non-functional, Dynamic, Manual\\
    
    Initial State: System is fully available and accessible to healthcare workers.\\
    
    Input/Condition: Healthcare workers perform key functions after a 30-minute training session.\\
    
    Output/Result: 90\% of healthcare workers complete tasks without assistanceand within the expected time frame.\\
    
    How this test will be performed: A test group of healthcare workers and the supervisor will be given the system, and a set of key functions to perform after a 30-minute training. They will be asked to complete these functions independently. After they do this, they will be given a task completion survey. In the task completion survey, the supervisor will verify that 90\% of workers completed tasks without assistance.

    \item{UR2\\}
    
    Type: Non-functional, Dynamic, Manual\\
    
    Initial State: System is deployed and accessible.\\
    
    Input/Condition: Users navigate and explore the system features independently.\\
    
    Output/Result: Majority of users can locate core functions without additional guidance.\\
    
    How this test will be performed: A test group of users and the supervisor will be given the system, and a set of core functions to locate. They will be asked to find these functions without guidance. After they do this, they will be given a navigation survey. In the navigation survey, the supervisor will verify that users could locate core functions independently.
\end{enumerate}

\subsubsection{Performance Requirement(NFR3)}

\paragraph{Performance}

\begin{enumerate}
    \item{PR1\\}
    
    Type: Non-functional, Dynamic, Automated\\
    
    Initial State: System with voice transcription feature is active.\\
    
    Input/Condition: Input of a 30-second audio recording under typical clinic noise levels.\\
    
    Output/Result: System completes transcription within 30 seconds with an accuracy of 85\%.\\
    
    How this test will be performed: A test group of users and the supervisor will be given the system, and a set of 30-second audio recordings with typical clinic noise. They will be asked to process these recordings through the transcription system. After they do this, they will be given a performance survey. In the performance survey, the supervisor will verify the transcription speed and accuracy meet the 85\% standard.

    \item{PR2\\}
    
    Type: Non-functional, Dynamic, Manual\\
    
    Initial State: Transcription interface open and ready.\\
    
    Input/Condition: Real-time voice input provided by healthcare professional.\\
    
    Output/Result: Real-time transcription displayed within a 1 second delay.\\
    
    How this test will be performed: A test group of healthcare professionals and the supervisor will be given the system, and a set of voice inputs to transcribe. They will be asked to speak these inputs for real-time transcription. After they do this, they will be given a latency survey. In the latency survey, the supervisor will verify that transcription delay remains under 1 second.
\end{enumerate}

\subsubsection{Operational Requirement(NFR4)}

\paragraph{Operational Requirement}

\begin{enumerate}
    \item{OR1\\}
    
    Type: Non-functional, Dynamic, Automated\\
    
    Initial State: System is live and connected to monitoring software.\\
    
    Input/Condition: 30-day operational period with intermittent load testing.\\
    
    Output/Result: Uptime is consistently maintained at 99.9\% or above.\\
    
    How test will be performed: Use uptime monitoring tools to track server availability over the 30-day period.

    \item{OR2\\}
    
    Type: Non-functional, Dynamic, Manual\\
    
    Initial State: System in operational use.\\
    
    Input/Condition: Users access the system over a 30-day period under normal and peak loads.\\
    
    Output/Result: Log data and user feedback confirm uptime meets standards with no significant disruptions.\\
    
    How this test will be performed: A test group of users and the supervisor will be given the system, and a set of usage patterns to follow over 30 days. They will be asked to use the system according to these patterns while monitoring performance. After they do this, they will be given an uptime survey. In the uptime survey, the supervisor will verify that no significant disruptions occurred during the test period.
\end{enumerate}

\subsubsection{Maintainability Requirement(NFR5)}

\paragraph{Maintainability Requirement}

\begin{enumerate}
    \item{MR1\\}
    
    Type: Non-functional, Dynamic, Manual\\
    
    Initial State: System running on the latest version with recent update logs.\\
    
    Input/Condition: Regular software update is applied for bug fixes and improvements.\\
    
    Output/Result: System successfully applies updates without impacting stability.\\
    
    How this test will be performed: A test group of users and the supervisor will be given the system, and a set of software updates to apply. They will be asked to implement these updates following standard procedures. After they do this, they will be given a stability survey. In the stability survey, the supervisor will verify that system functionality remained stable after updates.


    \item{MR2\\}
    
    Type: Non-functional, Dynamic, Manual\\

    Initial State: Previous version of the system with identified bugs or issues.\\

    Input/Condition: Apply updates addressing known issues.\\

    Output/Result: System functions as expected with resolved issues and no new errors introduced.\\

    How this test will be performed: While developers update the system, a test group of users and a supervisor will monitor it. The group will test functionality as updates are deployed, after which they'll complete a regression survey. The supervisor will verify that issues are resolved without introducing new errors.
\end{enumerate}

\subsubsection{Security Requirement(NFR6)}

\paragraph{Security Requirement}

\begin{enumerate}
    \item{SR1\\}
    
    Type: Non-functional, Dynamic, Automated\\
    
    Initial State: System with live patient data encryption protocols active.\\
    
    Input/Condition: Security audit test is performed on the system.\\
    
    Output/Result: No vulnerabilities are detected; all data remains encrypted in transit and at rest.\\
    
    How test will be performed: Conduct automated vulnerability scans and manual security audit to confirm full compliance with PIPEDA and encryption standards.

    \item{SR2\\}
    
    Type: Non-functional, Dynamic, Manual\\
    
    Initial State: System fully operational with access logs enabled.\\
    
    Input/Condition: Simulate unauthorized access attempts.\\
    
    Output/Result: Unauthorized attempts are blocked; access logs capture details.\\
    
    How this test will be performed: A test group of users and the supervisor will be given the system, and a set of unauthorized access scenarios. They will be asked to attempt these scenarios while monitoring system responses. After they do this, they will be given an access control survey. In the access control survey, the supervisor will verify that all unauthorized attempts were properly blocked and logged.
\end{enumerate}

\subsubsection{Cultural Requirement (NFR7)}

\paragraph{Language Customization}

\begin{enumerate}
    \item{CR1\\}
    
    Type: Non-functional, Dynamic, Manual\\
    
    Initial State: System operational in default language (English).\\
    
    Input/Condition: User selects an alternate language from the settings menu.\\
    
    Output/Result: System displays all content in the selected language without loss of functionality.\\
    
    How this test will be performed: A test group of users and the supervisor will be given the system, and a set of language switching scenarios. They will be asked to change the system language and verify content display. After they do this, they will be given a language functionality survey. In the language functionality survey, the supervisor will verify that all content displays correctly in each language.

    \item{CR2\\}
    
    Type: Non-functional, Dynamic, Manual\\
    
    Initial State: System set to the default language with alternative language packs installed.\\

    Input/Condition: User navigates through various sections in alternative language selected.\\
    
    Output/Result: No functional errors or misaligned text appear in the UI.\\
    
    How this test will be performed: A test group of users and the supervisor will be given the system, and a set of UI navigation tasks in different languages. They will be asked to complete these tasks in various language settings. After they do this, they will be given a language consistency survey. In the language consistency survey, the supervisor will verify that no text misalignment or functional errors occur.

\end{enumerate}

\subsubsection{Legal Requirement(NFR8)}

\paragraph{Legal Compliance}

\begin{enumerate}
    \item{LR1\\}
    
    Type: Non-functional, Dynamic, Manual\\
    
    Initial State: System is fully functional and contains patient records.\\
    
    Input/Condition: Conduct a compliance audit against PIPEDA and other relevant data protection standards.\\
    
    Output/Result: System passes all compliance checks with no exceptions.\\
    
    How test will be performed: A certified compliance auditor verifies data handling, encryption protocols, and data access policies.

    \item{LR2\\}
    
    Type: Non-functional, Dynamic, Automated\\
    
    Initial State: System storing and transmitting patient data over a network.\\
    
    Input/Condition: Monitor data handling and transfer processes during operation.\\
    
    Output/Result: All patient data is handled in compliance with regulations, without any unauthorized access or data breaches.\\
    
    How test will be performed: Automated compliance tools track data handling practices over a defined period, and alerts are set up for any regulatory deviations.
\end{enumerate}

\subsubsection{Scalability (NFR9)}

\paragraph{Scalability}

\begin{enumerate}

    \item{S1\\}
    
    Type: Non-functional, Dynamic, Automated\\
    
    Initial State: System deployed on a test server environment capable of scaling horizontally.\\
    
    Input/Condition: Simulate an increasing number of concurrent users accessing the system, starting from 100 users up to 10,000 users.\\
    
    Output/Result: System maintains consistent performance and response times without a drop in performance.\\
    
    How test will be performed: Use load testing tools like Apache to simulate concurrent user traffic and monitor system response times, server load, and throughput during the test.
    
    \item{S2\\}
    
    Type: Non-functional, Dynamic, Automated\\
    
    Initial State: System operational with data processing services enabled.\\
    
    Input/Condition: Increase the data input rate gradually from standard load to peak operational load.\\
    
    Output/Result: System processes data without bottlenecks, maintaining efficient load distribution.\\
    
    How this test will be performed: Conduct automated stress tests with data processing simulators and monitor resource usage, CPU, memory, and load balancer performance to ensure scalability standards are met.
    
\end{enumerate}

\subsection{Tests for Safety and Security Requirements}

\subsubsection{Access Requirements Tests}
\begin{enumerate}
    \item{AC1-1\\}
    
    Type: Functional, Dynamic, Automated\\
    
    Initial State: System deployed with authentication module enabled and test user accounts configured.\\
    
    Input/Condition: Attempt to access protected resources with invalid credentials 5 times consecutively from the same IP address.\\
    
    Output/Result: 
    \begin{itemize}
        \item System logs each failed attempt
        \item Account is temporarily locked after 5 failed attempts
        \item Security team receives notification
        \item User receives lockout notification
    \end{itemize}
    
    How this test will be performed: A test group of users and the supervisor will be given the system, and a set of sample invalid credentials for the system. They will be asked to use these credentials repeatedly to attempt system access. After they do this, they will be given a security test survey. In the security test survey, the supervisor will verify the logging, lockout, and notification requirements are met.

    \item{AC1-2\\}
    
    Type: Functional, Dynamic, Manual\\
    
    Initial State: System operational with authentication logs enabled.\\
    
    Input/Condition: Attempt to access system resources without authentication.\\
    
    Output/Result: 
    \begin{itemize}
        \item All unauthorized access attempts are blocked
        \item Each attempt is logged with timestamp, IP address, and attempted resource
    \end{itemize}
    
    How this test will be performed: A test group of users and the supervisor will be given the system, and a set of protected resources to access. They will be asked to attempt accessing these resources without authentication. After they do this, they will be given a security test survey. In the security test survey, the supervisor will verify that access was properly blocked and logged.


    \item{AC2-1\\}
    
    Type: Functional, Dynamic, Manual\\
    
    Initial State: System operational with standard user and admin accounts configured.\\
    
    Input/Condition: Attempt to create, update, and delete user accounts using non-admin credentials.\\
    
    Output/Result:
    \begin{itemize}
        \item All unauthorized actions are blocked
        \item Actions are logged with user details
        \item Security team can review blocked attempts
    \end{itemize}
    
    How this test will be performed: A test group of users and the supervisor will be given the system, and a set of standard user credentials. They will be asked to attempt various administrative operations using these credentials. After they do this, they will be given a security test survey. In the security test survey, the supervisor will verify that administrative actions were properly restricted and logged.

\end{enumerate}

\subsubsection{Integrity Requirements Tests}
\begin{enumerate}
    \item{IR1-1\\}
    
    Type: Non-functional, Dynamic, Automated\\
    
    Initial State: System operational with test user accounts and authentication database.\\
    
    Input/Condition: Simulate multiple concurrent failed login attempts while monitoring credential storage.\\
    
    Output/Result: User credentials remain unchanged and system maintains stability.\\
    
    How this test will be performed: A test group of users and the supervisor will be given the system, and a set of test credentials for concurrent authentication testing. They will be asked to perform multiple simultaneous authentication attempts. After they do this, they will be given a system integrity survey. In the system integrity survey, the supervisor will verify that all credentials remained intact and system stability was maintained.

\end{enumerate}

\newpage

\begin{landscape}
\subsection{Traceability Between Test Cases and Requirements}
\begingroup
\setlength{\tabcolsep}{4pt}  % Reduce column spacing
\begin{longtable}{|p{1.1cm}|p{0.85cm}|p{0.85cm}|p{0.85cm}|p{0.85cm}|p{0.85cm}|p{0.85cm}|p{0.85cm}|p{0.85cm}|p{0.85cm}|p{0.85cm}|p{0.85cm}|p{0.85cm}|p{0.85cm}|p{0.85cm}|p{0.85cm}|p{0.85cm}|p{0.6cm}|p{0.6cm}|}
\hline
\textbf{NFR ID} & \textbf{AD1} & \textbf{AD2} & \textbf{UR1} & \textbf{UR2} & \textbf{PR1} & \textbf{PR2} & \textbf{OR1} & \textbf{OR2} & \textbf{MR1} & \textbf{MR2} & \textbf{SR1} & \textbf{SR2} & \textbf{CR1} & \textbf{CR2} & \textbf{LR1} & \textbf{LR2} & \textbf{S1} & \textbf{S2} \\
\hline
\endfirsthead
\multicolumn{19}{c}{Table \ref{tab:nfr-test-traceability} continued} \\
\hline
\textbf{NFR ID} & \textbf{AD1} & \textbf{AD2} & \textbf{UR1} & \textbf{UR2} & \textbf{PR1} & \textbf{PR2} & \textbf{OR1} & \textbf{OR2} & \textbf{MR1} & \textbf{MR2} & \textbf{SR1} & \textbf{SR2} & \textbf{CR1} & \textbf{CR2} & \textbf{LR1} & \textbf{LR2} & \textbf{S1} & \textbf{S2} \\
\hline
\endhead
\hline
\multicolumn{19}{r}{Continued on next page} \\
\endfoot
\hline
\caption{Non-Functional Requirements Test Cases Traceability Matrix} \label{tab:nfr-test-traceability}
\endlastfoot
NFR1 & $\times$ & $\times$ & & & & & & & & & & & & & & & & \\
\hline
NFR2 & & & $\times$ & $\times$ & & & & & & & & & & & & & & \\
\hline
NFR3 & & & & & $\times$ & $\times$ & & & & & & & & & & & & \\
\hline
NFR4 & & & & & & & $\times$ & $\times$ & & & & & & & & & & \\
\hline
NFR5 & & & & & & & & & $\times$ & $\times$ & & & & & & & & \\
\hline
NFR6 & & & & & & & & & & & $\times$ & $\times$ & & & & & & \\
\hline
NFR7 & & & & & & & & & & & & & $\times$ & $\times$ & & & & \\
\hline
NFR8 & & & & & & & & & & & & & & & $\times$ & $\times$ & & \\
\hline
NFR9 & & & & & & & & & & & & & & & & & $\times$ & $\times$ \\
\end{longtable}
\endgroup
\end{landscape}
\newpage

\subsection{Unit Testing Scope}

\wss{What modules are outside of the scope.  If there are modules that are
  developed by someone else, then you would say here if you aren't planning on
  verifying them.  There may also be modules that are part of your software, but
  have a lower priority for verification than others.  If this is the case,
  explain your rationale for the ranking of module importance.}

\subsection{Tests for Functional Requirements}

\wss{Most of the verification will be through automated unit testing.  If
  appropriate specific modules can be verified by a non-testing based
  technique.  That can also be documented in this section.}

\subsubsection{Module 1}

\wss{Include a blurb here to explain why the subsections below cover the module.
  References to the MIS would be good.  You will want tests from a black box
  perspective and from a white box perspective.  Explain to the reader how the
  tests were selected.}

\begin{enumerate}

\item{test-id1\\}

Type: \wss{Functional, Dynamic, Manual, Automatic, Static etc. Most will
  be automatic}
					
Initial State: 
					
Input: 
					
Output: \wss{The expected result for the given inputs}

Test Case Derivation: \wss{Justify the expected value given in the Output field}

How test will be performed: 
					
\item{test-id2\\}

Type: \wss{Functional, Dynamic, Manual, Automatic, Static etc. Most will
  be automatic}
					
Initial State: 
					
Input: 
					
Output: \wss{The expected result for the given inputs}

Test Case Derivation: \wss{Justify the expected value given in the Output field}

How test will be performed: 

\item{...\\}
    
\end{enumerate}

\subsubsection{Module 2}

...
				
\bibliographystyle{plainnat}

\bibliography{../../refs/References}

\newpage

\section{Appendix}

This is where you can place additional information.

\subsection{Symbolic Parameters}

The definition of the test cases will call for SYMBOLIC\_CONSTANTS.
Their values are defined in this section for easy maintenance.

\subsection{Usability Survey Questions?}

\wss{This is a section that would be appropriate for some projects.}

\newpage{}
\section*{Appendix --- Reflection}

\wss{This section is not required for CAS 741}

The information in this section will be used to evaluate the team members on the
graduate attribute of Lifelong Learning.

The purpose of reflection questions is to give you a chance to assess your own
learning and that of your group as a whole, and to find ways to improve in the
future. Reflection is an important part of the learning process.  Reflection is
also an essential component of a successful software development process.  

Reflections are most interesting and useful when they're honest, even if the
stories they tell are imperfect. You will be marked based on your depth of
thought and analysis, and not based on the content of the reflections
themselves. Thus, for full marks we encourage you to answer openly and honestly
and to avoid simply writing ``what you think the evaluator wants to hear.''

Please answer the following questions.  Some questions can be answered on the
team level, but where appropriate, each team member should write their own
response:


\begin{enumerate}
  \item What went well while writing this deliverable? 
  \item What pain points did you experience during this deliverable, and how
    did you resolve them?
  \item What knowledge and skills will the team collectively need to acquire to
  successfully complete the verification and validation of your project?
  Examples of possible knowledge and skills include dynamic testing knowledge,
  static testing knowledge, specific tool usage, Valgrind etc.  You should look to
  identify at least one item for each team member.
  \item For each of the knowledge areas and skills identified in the previous
  question, what are at least two approaches to acquiring the knowledge or
  mastering the skill?  Of the identified approaches, which will each team
  member pursue, and why did they make this choice?
\end{enumerate}

\end{document}