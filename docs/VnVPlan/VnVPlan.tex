\documentclass[12pt, titlepage]{article}

\usepackage{booktabs}
\usepackage{longtable}
\usepackage{pdflscape}
\usepackage{tabularx}
\usepackage{hyperref}
\hypersetup{
    colorlinks,
    citecolor=blue,
    filecolor=black,
    linkcolor=red,
    urlcolor=blue
}
\usepackage[round]{natbib}

%% Comments

\usepackage{color}

\newif\ifcomments\commentstrue %displays comments
%\newif\ifcomments\commentsfalse %so that comments do not display

\ifcomments
\newcommand{\authornote}[3]{\textcolor{#1}{[#3 ---#2]}}
\newcommand{\todo}[1]{\textcolor{red}{[TODO: #1]}}
\else
\newcommand{\authornote}[3]{}
\newcommand{\todo}[1]{}
\fi

\newcommand{\wss}[1]{\authornote{blue}{SS}{#1}} 
\newcommand{\plt}[1]{\authornote{magenta}{TPLT}{#1}} %For explanation of the template
\newcommand{\an}[1]{\authornote{cyan}{Author}{#1}}

%% Common Parts

\newcommand{\progname}{ProgName} % PUT YOUR PROGRAM NAME HERE
\newcommand{\authname}{Team \#, Team Name
\\ Student 1 name
\\ Student 2 name
\\ Student 3 name
\\ Student 4 name} % AUTHOR NAMES                  

\usepackage{hyperref}
    \hypersetup{colorlinks=true, linkcolor=blue, citecolor=blue, filecolor=blue,
                urlcolor=blue, unicode=false}
    \urlstyle{same}
                                


\begin{document}

\title{System Verification and Validation Plan for \progname{}} 
\author{\authname}
\date{\today}
	
\maketitle

\pagenumbering{roman}

\section*{Revision History}

\begin{tabularx}{\textwidth}{p{3cm}p{2cm}X}
\toprule {\bf Date} & {\bf Version} & {\bf Notes}\\
\midrule
02-11-2024 & Gurleen Rahi & Summary, Objectives, Challenge Level and Extras, Relevant Documentation\\
03-11-2024 & Inreet Kaur & Tests for functional requirements and safety and security requirements, traceability matrix\\
03-11-2024 & Inreet Kaur & Verification and Validation team,SRS Verification Plan, Design Verification Plan\\

\bottomrule
\end{tabularx}

~\\

\newpage

\tableofcontents

\listoftables
\wss{Remove this section if it isn't needed}

\listoffigures
\wss{Remove this section if it isn't needed}

\newpage

\section{Symbols, Abbreviations, and Acronyms}

\renewcommand{\arraystretch}{1.2}
\begin{tabular}{l l} 
  \toprule		
  \textbf{symbol} & \textbf{description}\\
  \midrule 
  SRS & Software Requirements Specification\\
  MG & Module Guide\\
  MIS & Module Interface Specification\\
  FR & Functional Requirements\\
  NFR & Non-Functional Requirements\\
  \bottomrule
\end{tabular}\\

% \wss{symbols, abbreviations, or acronyms --- you can simply reference the SRS
%   \citep{SRS} tables, if appropriate}

\newpage

\pagenumbering{arabic}

\section{General Information}

\subsection{Summary}

This document provides a comprehensive description of the system tests specific to all requirements for a software application, RapidCare, that aims to streamline the healthcare documentation process aimed to be run as a web application. This document will be used as a verification tool to ensure that system fulfills all the requirements and needs of the user. This document will allow for an in-depth description of system tests for functional, non-functional, safety and security requirements. Additionally, it will outline the verification plan for design and implementation which will make sure that all requirements are fulfilled. Along with this, the document will also include the software validation plan that will validate the requirements by making sure that the client who intends to use this application is satisfied with its features.    

\subsection{Objectives}

The objective of verification and validation plan is to build confidence in the correctness of the software and decide if the software is correctly following the requirements. Additionally, we want to achieve adequate usability of our software by making sure it satisfies with the user's needs, thus speeding up the documentation process. Another goal of this document is to test the accuracy of the input data to reflect that the software suggests correct diagnosis based on the written text that has been transcribed from the audio conversation. Assuming that the external libraries are already tested by their implementation team, the priority of this document will be to test the requirements of the software to ease the patient care. Due to time constraints and limited resources, we are not going to test the quality of usability. While the usability is important to test, the priority of this document is to test the fundamental requirements to ensure the software is accuarate and reliable. 

\subsection{Challenge Level and Extras}

In terms of the project challenge level, the project will come in the general category. The reasoning for this choice is supported below:

\begin{itemize}
  \item\textbf{Domain Knowledge}-- The documentation process has a lot of ins and outs which may differ between health organizations. Our supervisors and stakeholders will provide us with a base of the requirements, but further elicitation will be required to ensure that the requirements reflect a problem that truly exists. Additionally, since the whole patient journey is tracked we will have to survey other hospital staff as well to gain a further understanding. 
  \item\textbf{Implementation Challenges}-- There will be quite a few microservices required for this project where each microservice has high complexity. The performance of the microservices must be high as this use case requires quick response time. Additionally, since we are dealing with patient data security and privacy must be upheld. Lastly, the integration between all of the parts must be secure and undergo rigorous integration testing.
\end{itemize}

As part of the extras for this project, we will accomplish the following extras:

\begin{itemize}
  \item\textbf{Usability testing}-- This is designed for the users to assess how easily they can navigate and use the software. This will ensure that the requirements of the software perfectly align with the user's needs.
  \item\textbf{User documentation}-- This will include instructions that will help the user to get started with the software. It will have an overview of the software along with some help resources for setup instructions and easy navigation for the user.
\end{itemize}

\subsection{Relevant Documentation}

The documents that are relevant to this project are \href{https://github.com/Inreet-Kaur/capstone/blob/main/docs/SRS/SRS.pdf}{SRS} \citep{SRS}, Hazard Analysis, MG and MIS. SRS and Hazard Analysis documents lists the FR, NFR, safety and security requirements which will assist us in developing tests for each of them. Moreover, MG and MIS documents give a structured approach of system's architecture and interface, ensuring that the system is thoroughly tested. This will help us to develop test cases for vital areas of the software.  


\section{Plan}

\wss{Introduce this section.  You can provide a roadmap of the sections to come.}

\subsection{Verification and Validation Team}

\wss{Your teammates.  Maybe your supervisor.
  You should do more than list names.  You should say what each person's role is
  for the project's verification.  A table is a good way to summarize this information.}

\subsection{SRS Verification Plan}

\wss{List any approaches you intend to use for SRS verification.  This may
  include ad hoc feedback from reviewers, like your classmates (like your
  primary reviewer), or you may plan for something more rigorous/systematic.}

\wss{If you have a supervisor for the project, you shouldn't just say they will
read over the SRS.  You should explain your structured approach to the review.
Will you have a meeting?  What will you present?  What questions will you ask?
Will you give them instructions for a task-based inspection?  Will you use your
issue tracker?}

\wss{Maybe create an SRS checklist?}

\subsection{Design Verification Plan}

\wss{Plans for design verification}

\wss{The review will include reviews by your classmates}

\wss{Create a checklists?}

\subsection{Verification and Validation Plan Verification Plan}

\wss{The verification and validation plan is an artifact that should also be
verified.  Techniques for this include review and mutation testing.}

\wss{The review will include reviews by your classmates}

\wss{Create a checklists?}

\subsection{Implementation Verification Plan}

\wss{You should at least point to the tests listed in this document and the unit
  testing plan.}

\wss{In this section you would also give any details of any plans for static
  verification of the implementation.  Potential techniques include code
  walkthroughs, code inspection, static analyzers, etc.}

\wss{The final class presentation in CAS 741 could be used as a code
walkthrough.  There is also a possibility of using the final presentation (in
CAS741) for a partial usability survey.}

\subsection{Automated Testing and Verification Tools}

\wss{What tools are you using for automated testing.  Likely a unit testing
  framework and maybe a profiling tool, like ValGrind.  Other possible tools
  include a static analyzer, make, continuous integration tools, test coverage
  tools, etc.  Explain your plans for summarizing code coverage metrics.
  Linters are another important class of tools.  For the programming language
  you select, you should look at the available linters.  There may also be tools
  that verify that coding standards have been respected, like flake9 for
  Python.}

\wss{If you have already done this in the development plan, you can point to
that document.}

\wss{The details of this section will likely evolve as you get closer to the
  implementation.}

\subsection{Software Validation Plan}

\wss{If there is any external data that can be used for validation, you should
  point to it here.  If there are no plans for validation, you should state that
  here.}

\wss{You might want to use review sessions with the stakeholder to check that
the requirements document captures the right requirements.  Maybe task based
inspection?}

\wss{For those capstone teams with an external supervisor, the Rev 0 demo should 
be used as an opportunity to validate the requirements.  You should plan on 
demonstrating your project to your supervisor shortly after the scheduled Rev 0 demo.  
The feedback from your supervisor will be very useful for improving your project.}

\wss{For teams without an external supervisor, user testing can serve the same purpose 
as a Rev 0 demo for the supervisor.}

\wss{This section might reference back to the SRS verification section.}

\section{System Tests} \label{section:4}

% \wss{There should be text between all headings, even if it is just a roadmap of
% the contents of the subsections.}
This section will include system tests for functional, non-functional, and safety and security requirements. In addition to this, we will include a traceability table for test cases and requirements.

\subsection{Tests for Functional Requirements} \label{section:4.1}

% \wss{Subsets of the tests may be in related, so this section is divided into
%   different areas.  If there are no identifiable subsets for the tests, this
%   level of document structure can be removed.}

% \wss{Include a blurb here to explain why the subsections below
%   cover the requirements.  References to the SRS would be good here.}

This section contains the tests for the Functional Requirements. The subsections for these tests were created based on the subsections of the Functional Requirements listed in the \href{https://github.com/Inreet-Kaur/capstone/blob/main/docs/SRS/SRS.pdf}{SRS} \citep{SRS}. Traceability for these requirements and tests can be found in the section (\ref{section:4.4}).


\subsubsection{Add a document to the database} \label{section:4.1.1}

This subsection covers FR1, FR4, and FR8 from of the \href{https://github.com/Inreet-Kaur/capstone/blob/main/docs/SRS/SRS.pdf}{SRS document} \citep{SRS} by testing that the system is able to add a document to the database only when a valid input is provided.

\begin{enumerate}

\item{test-FR1,4,8-1} \label{test-FR1,4,8-1}

Control: Manual

Initial State: The system is set up, open to the relevant section (e.g., Healthcare Network, Healthcare Professional, or Patient Records), and ready to take the user input.

Input: Correct and complete input data for all required fields.

Output: A confirmation message and a new entry is added to the appropriate database.

Test Case Derivation: The system will validate the input data, accept a complete and correct data input, and confirm a successful addition to the database.

How test will be performed: The test controller will input a valid data object and check if the system is able to validate and accept a valid input. The controller will verify that a confirmation message appears and there is a valid new entry in the appropriate database.

					
\item{test-FR1,4,8-2} \label{test-FR1,4,8-2}

Control: Manual
					
Initial State: The system is set up, open to the relevant section (e.g., Healthcare Network, Healthcare Professional, or Patient Records), and ready to take the user input.

Input: Invalid input data.

Output: An appropriate error message.

Test Case Derivation: The system will validate the input data and prompt an error message outlining why the system is not able to accept the input data. No new document is added to the database.

How test will be performed: The test controller will input an invalid data object and check if the system is able to validate and reject the invalid input. The controller will also verify that an error message appears and there is no new entry in the appropriate database.

\end{enumerate}


\subsubsection{Remove a document from the database} \label{section:4.1.2}

This subsection covers FR2, FR5, and FR9 from of the \href{https://github.com/Inreet-Kaur/capstone/blob/main/docs/SRS/SRS.pdf}{SRS document} \citep{SRS} by testing that the system is able to remove a document to the database only when a valid input idetifier is provided.

\begin{enumerate}

\item{test-FR2,5,9-1} \label{test-FR2,5,9-1}

Control: Manual

Initial State: The system is set up, open to the relevant section (e.g., Healthcare Network, Healthcare Professional, or Patient Records), and ready to take the user input. A document exists in the appropriate database.

Input: A correct identifier for the document to be deleted.

Output: A confirmation message and relevant entry no longer exist in the database.

Test Case Derivation: The system should allow the deletion of an existing document when the correct identifier is provided. 

How test will be performed: The test controller will input a valid identifier and check if the system is able to validate and accept a valid input. The controller will verify that a confirmation message appears and there the associated document no longer exists in the database.

					
\item{test-FR2,5,9-2} \label{test-FR2,5,9-2}

Control: Manual

Initial State: The system is set up, open to the relevant section (e.g., Healthcare Network, Healthcare Professional, or Patient Records), and ready to take the user input. A document exists in the appropriate database.

Input: An invalid identifier for the document to be deleted.

Output: An error message.

Test Case Derivation: The system should be able to validate the provided identifier and prevent the deletion of any existing document. 

How test will be performed: The test controller will input an invalid identifier and check if the system is able to validate and reject the invalid identifier. The controller will verify that an error message appears, and no document is deleted from the database.

\end{enumerate}


\subsubsection{Update document in the database} \label{section:4.1.3}

\item{test-} \label{test-}


\subsubsection{login for valid/invalid credentials} \label{section:4.1.4}

\item{test-} \label{test-}


\subsubsection{Voice-to-text-transcription check} \label{section:4.1.5}

\item{test-} \label{test-}


\subsubsection{Validate output of correct diagonsis and medication} \label{section:4.1.6}

\item{test-} \label{test-}


\subsubsection{Validate input data for models} \label{section:4.1.7}

\item{test-} \label{test-}


\subsubsection{Verify completeness and correctness of data} \label{section:4.1.8}

\item{test-} \label{test-}




\subsection{Tests for Nonfunctional Requirements} \label{section:4.2}

\subsubsection{Aesthetic and Design (NFR1)} \label{section:4.2.1}

\begin{enumerate}
    \item{test-AD1\\} \label{test-AD1}
    
    Type: Non-functional, Dynamic, Manual\\
    
    Initial State: UI is designed and implemented.\\
    
    Input/Condition: Healthcare workers view the UI under normal operating conditions.\\
    
    Output/Result: Feedback collected on UI’s aesthetic appeal and simplicity.\\
    
    How test will be performed: Conduct a comprehensive survey with healthcare workers, collecting quantitative and qualitative responses on UI usability and design clarity.

    \item{test-AD2\\} \label{test-AD2}
    
    Type: Non-functional, Dynamic, Manual\\
    
    Initial State: UI is operational and accessible to users.\\
    
    Input/Condition: Users interact with the UI during routine tasks and provides feedback.\\
    
    Output/Result: 80\% of users report satisfaction with the UI design.\\
    
    How test will be performed: Conduct observational tests and collect detailed feedback to analyze user experience, ensuring primary functions are accessible within two clicks from the main screen. 
\end{enumerate}

\subsubsection{Usability Requirement(NFR2)} \label{section:4.2.2}

\begin{enumerate}
    \item{test-UR1\\} \label{test-UR1}
    
    Type: Non-functional, Dynamic, Manual\\
    
    Initial State: System is fully available and accessible to healthcare workers.\\
    
    Input/Condition: Healthcare workers perform key functions after a 30-minute training session.\\
    
    Output/Result: 90\% of healthcare workers complete tasks without assistanceand within the expected time frame.\\
    
    How test will be performed: Conduct structured observations, recording task completion rates and collecting user feedback on ease of use.

    \item{test-UR2\\} \label{test-UR2}
    
    Type: Non-functional, Dynamic, Manual\\
    
    Initial State: System is deployed and accessible.\\
    
    Input/Condition: Users navigate and explore the system features independently.\\
    
    Output/Result: Majority of users can locate core functions without additional guidance.\\
    
    How test will be performed: Use task-based usability tests and track completion times and feedback to confirm high detectability of system functions.
\end{enumerate}

\subsubsection{Performance Requirement(NFR3)} \label{section:4.2.3}

\begin{enumerate}
    \item{test-PR1\\} \label{test-PR1}
    
    Type: Non-functional, Dynamic, Automated\\
    
    Initial State: System with voice transcription feature is active.\\
    
    Input/Condition: Input of a 30-second audio recording under typical clinic noise levels.\\
    
    Output/Result: System completes transcription within 30 seconds with an accuracy of 85\%.\\
    
    How test will be performed: Use audio samples in a controlled setting to measure transcription speed and accuracy, ensuring the system meets performance standards.

    \item{test-PR2\\} \label{test-PR2}
    
    Type: Non-functional, Dynamic, Manual\\
    
    Initial State: Transcription interface open and ready.\\
    
    Input/Condition: Real-time voice input provided by healthcare professional.\\
    
    Output/Result: Real-time transcription displayed within a 1 second delay.\\
    
    How test will be performed: Conduct real-time testing and monitor the delay in transcription display, verifying it meets standards.
\end{enumerate}

\subsubsection{Operational Requirement(NFR4)} \label{section:4.2.4}

\begin{enumerate}
    \item{test-OR1\\} \label{test-OR1}
    
    Type: Non-functional, Dynamic, Automated\\
    
    Initial State: System is live and connected to monitoring software.\\
    
    Input/Condition: 30-day operational period with intermittent load testing.\\
    
    Output/Result: Uptime is consistently maintained at 99.9\% or above.\\
    
    How test will be performed: Use uptime monitoring tools to track server availability over the 30-day period.

    \item{test-OR2\\} \label{test-OR2}
    
    Type: Non-functional, Dynamic, Manual\\
    
    Initial State: System in operational use.\\
    
    Input/Condition: Users access the system over a 30-day period under normal and peak loads.\\
    
    Output/Result: Log data and user feedback confirm uptime meets standards with no significant disruptions.\\
    
    How test will be performed: Review system logs and usage reports at the end of the period to ensure compliance with operational goals.
\end{enumerate}

\subsubsection{Maintainability Requirement(NFR5)} \label{section:4.2.5}

\begin{enumerate}
    \item{test-MR1\\} \label{test-MR1}
    
    Type: Non-functional, Dynamic, Manual\\
    
    Initial State: System running on the latest version with recent update logs.\\
    
    Input/Condition: Regular software update is applied for bug fixes and improvements.\\
    
    Output/Result: System successfully applies updates without impacting stability.\\
    
    How test will be performed: Conduct a manual inspection of update logs and perform basic functionality checks after each update.

    \item{test-MR2\\} \label{test-MR2}
    
    Type: Non-functional, Dynamic, Manual\\

    Initial State: Previous version of the system with identified bugs or issues.\\

    Input/Condition: Apply updates addressing known issues.\\

    Output/Result: System functions as expected with resolved issues and no new errors introduced.\\

    How test will be performed: Conduct regression testing to verify that updates did not affect system performance.
\end{enumerate}

\subsubsection{Security Requirement(NFR6)} \label{section:4.2.6}

\begin{enumerate}
    \item{test-SR1\\} \label{test-SR1}
    
    Type: Non-functional, Dynamic, Automated\\
    
    Initial State: System with live patient data encryption protocols active.\\
    
    Input/Condition: Security audit test is performed on the system.\\
    
    Output/Result: No vulnerabilities are detected; all data remains encrypted in transit and at rest.\\
    
    How test will be performed: Conduct automated vulnerability scans and manual security audit to confirm full compliance with PIPEDA and encryption standards.

    \item{test-SR2\\} \label{test-SR2}
    
    Type: Non-functional, Dynamic, Manual\\
    
    Initial State: System fully operational with access logs enabled.\\
    
    Input/Condition: Simulate unauthorized access attempts.\\
    
    Output/Result: Unauthorized attempts are blocked; access logs capture details.\\
    
    How test will be performed: Attempt controlled security breaches to validate the integrity of access control and logging mechanisms.
\end{enumerate}

\subsubsection{Cultural Requirement (NFR7)} \label{section:4.2.7}

\begin{enumerate}
    \item{test-CR1\\} \label{test-CR1}
    
    Type: Non-functional, Dynamic, Manual\\
    
    Initial State: System operational in default language (English).\\
    
    Input/Condition: User selects an alternate language from the settings menu.\\
    
    Output/Result: System displays all content in the selected language without loss of functionality.\\
    
    How test will be performed: Users switch between languages and verify that UI elements and content display correctly in each selected language.

    \item{test-CR2\\} \label{test-CR2}
    
    Type: Non-functional, Dynamic, Manual\\
    
    Initial State: System set to the default language with alternative language packs installed.\\

    Input/Condition: User navigates through various sections in alternative language selected.\\
    
    Output/Result: No functional errors or misaligned text appear in the UI.\\
    
    How test will be performed: Conduct navigation tests across UI components in different languages to ensure consistent experience.
\end{enumerate}

\subsubsection{Legal Requirement(NFR8)} \label{section:4.2.8}

\begin{enumerate}
    \item{test-LR1\\} \label{test-LR1}
    
    Type: Non-functional, Dynamic, Manual\\
    
    Initial State: System is fully functional and contains patient records.\\
    
    Input/Condition: Conduct a compliance audit against PIPEDA and other relevant data protection standards.\\
    
    Output/Result: System passes all compliance checks with no exceptions.\\
    
    How test will be performed: A certified compliance auditor verifies data handling, encryption protocols, and data access policies.

    \item{test-LR2\\} \label{test-LR2}
    
    Type: Non-functional, Dynamic, Automated\\
    
    Initial State: System storing and transmitting patient data over a network.\\
    
    Input/Condition: Monitor data handling and transfer processes during operation.\\
    
    Output/Result: All patient data is handled in compliance with regulations, without any unauthorized access or data breaches.\\
    
    How test will be performed: Automated compliance tools track data handling practices over a defined period, and alerts are set up for any regulatory deviations.
\end{enumerate}

\subsubsection{Scalability (NFR9)} \label{section:4.2.9}

\begin{enumerate}

    \item{test-S1\\} \label{test-S1}
    
    Type: Non-functional, Dynamic, Automated\\
    
    Initial State: System deployed on a test server environment capable of scaling horizontally.\\
    
    Input/Condition: Simulate an increasing number of concurrent users accessing the system, starting from 100 users up to 10,000 users.\\
    
    Output/Result: System maintains consistent performance and response times without a drop in performance.\\
    
    How test will be performed: Use load testing tools liek Apache to simulate concurrent user traffic and monitor system response times, server load, and throughput during the test.
    
    \item{test-S2\\} \label{test-S2}
    
    Type: Non-functional, Dynamic, Automated\\
    
    Initial State: System operational with data processing services enabled.\\
    
    Input/Condition: Increase the data input rate gradually from standard load to peak operational load.\\
    
    Output/Result: System processes data without bottlenecks, maintaining efficient load distribution.\\
    
    How test will be performed: Conduct stress tests with data processing simulators and monitor resource usage, CPU, memory, and load balancer performance to ensure scalability standards are met.
    
\end{enumerate}


\subsection{Tests for Safety and Security Requirements} \label{section:4.3}

\subsubsection{Access Requirements Tests} \label{section:4.3.1}

\begin{enumerate}

    \item{test-AC1-1\\} \label{test-AC1-1}
    
    Type: Functional, Dynamic, Automated\\
    
    Initial State: System deployed with authentication module enabled and test user accounts configured.\\
    
    Input/Condition: Attempt to access protected resources with invalid credentials 5 times consecutively from the same IP address.\\
    
    Output/Result: 
    \begin{itemize}
        \item System logs each failed attempt
        \item Account is temporarily locked after 5 failed attempts
        \item Security team receives notification
        \item User receives lockout notification
    \end{itemize}
    
    How test will be performed: Use automated testing scripts to simulate multiple failed login attempts and verify system responses, log entries, and notification delivery.

    \item{test-AC1-2\\}  \label{test-AC1-2}
    
    Type: Functional, Dynamic, Manual\\
    
    Initial State: System operational with authentication logs enabled.\\
    
    Input/Condition: Attempt to access system resources without authentication.\\
    
    Output/Result: 
    \begin{itemize}
        \item All unauthorized access attempts are blocked
        \item Each attempt is logged with timestamp, IP address, and attempted resource
    \end{itemize}
    
    How test will be performed: Manual testing of various system endpoints and resources without authentication tokens.

    \item{test-AC2-1\\}  \label{test-AC2-1}
    
    Type: Functional, Dynamic, Manual\\
    
    Initial State: System operational with standard user and admin accounts configured.\\
    
    Input/Condition: Attempt to create, update, and delete user accounts using non-admin credentials.\\
    
    Output/Result:
    \begin{itemize}
        \item All unauthorized actions are blocked
        \item Actions are logged with user details
        \item Security team can review blocked attempts
    \end{itemize}
    
    How test will be performed: Manual testing using different user account types to attempt administrative actions.
\end{enumerate}


\subsubsection{Integrity Requirements Tests} \label{section:4.3.2}

\begin{enumerate}
    \item{test-IR1-1\\}  \label{test-IR1-1}
    
    Type: Non-functional, Dynamic, Automated\\
    
    Initial State: System operational with test user accounts and authentication database.\\
    
    Input/Condition: Simulate multiple concurrent failed login attempts while monitoring credential storage.\\
    
    Output/Result: User credentials remain unchanged and system maintains stability.\\
    
    How test will be performed: Use load testing tools to simulate concurrent authentication attempts while monitoring database integrity.


    \item{test-IR2-1}: Generate error Messages \label{test-IR2-1}
    
    Type: Functional, Dynamic, Manual\\
    
    Initial State: The system is set up, open to the relevant section and ready to take the user input.\\
    
    Input/Condition: Submit an invalid data input.\\
    
    Output/Result: An error message is displayed to the user. No data is added to any database. 
    
    How test will be performed: The test controller will input an invalid data input and check if the system is able to validate and reject the invalid data. The controller will verify that an error message appears.


    \item{test-IR3-1\\}: Validation of confidence score (Pranav) \label{test-IR3-1}
    
    % Type: Functional, Static, Automated\\
    
    % Initial State: ML model trained and ready for predictions.\\
    
    % Input/Condition: Submit test dataset for predictions.\\
    
    % Output/Result: 
    % \begin{itemize}
    %     \item Each prediction includes a confidence score
    %     \item Validation accuracy scores are generated after training
    %     \item Scores are properly formatted and within expected ranges
    % \end{itemize}
    
    % How test will be performed: Automated testing of model outputs and validation metrics.

    \item{test-IR4-1}: Duplicate record detection \label{test-IR4-1}
    
    Type: Functional, Dynamic, Manual\\
    
    Initial State: The system is set up, open to the relevant section and ready to take the user input. A document already exists in the relevant database.\\
    
    Input/Condition: A new input with the same data as an existing document.\\
    
    Output/Result: An error message. No new document is added to the database.\\
    
    How test will be performed: The test controller will input a new input with the same data as an existing document and check if the system is able to validate and reject the duplicate record. The controller will verify that an error message appears.

    % \item{IR5-1\\}
    
    % Type: Functional, Static, Automated\\
    
    % Initial State: Test dataset with mix of relevant and irrelevant parameters.\\
    
    % Input/Condition: Process test dataset through model input pipeline.\\
    
    % Output/Result: Only relevant parameters are passed to the model for processing.\\
    
    % How test will be performed: Automated verification of model input parameters against predefined parameter list.

    \item{test-IR6-1\\}: Classification of data for report generation. (Gurleen) \label{test-IR6-1}
    
    % Type: Functional, Dynamic, Automated\\
    
    % Initial State: System configured with report generation capabilities.\\
    
    % Input/Condition: Generate reports with various data classifications.\\
    
    % Output/Result: Reports are generated with correct classification and validation checks passed.\\
    
    % How test will be performed: Automated testing of report generation with predefined test cases.

    % \item{IR7-1\\}
    
    % Type: Functional, Dynamic, Automated\\
    
    % Initial State: Audio transcription system enabled with test audio files.\\
    
    % Input/Condition: Process audio files with varying levels of background noise.\\
    
    % Output/Result: Transcribed data accurately reflects spoken content with minimal noise interference.\\
    
    % How test will be performed: Automated testing using pre-recorded audio samples with known content and noise levels.

\end{enumerate}

\newpage


\begin{landscape}
\subsection{Traceability Between Test Cases and Requirements} \label{section:4.4}

\begin{table}[H]
  \centering
  \begin{tabular}{|c|c|c|c|c|c|c|c|c|c|c|c|c|c|c|}
  \hline
   Test ID & FR1 & FR2 & FR3 & FR4 & FR5 & FR6 & FR7 & FR8 & FR9 & FR10 & FR11 & FR12 & FR13 & FR14\\
  \hline
  test-FR1,4,8-\ref{test-FR1,4,8-1} & $\times$ & & & $\times$ & & & & $\times$ & & & & & & \\
  \hline
  test-FR1,4,8-\ref{test-FR1,4,8-2} & $\times$ & & & $\times$ & & & & $\times$ & & & & & & \\
  \hline
  test-FR2,5,9-\ref{test-FR2,5,9-1} & & $\times$ & & & $\times$ & & & & $\times$ & & & & & \\
  \hline
  test-FR2,5,9-\ref{test-FR2,5,9-2} & & $\times$ & & & $\times$ & & & & $\times$ & & & & & \\
  \hline
  test--\ref{test-} & & & & & & & & & & & & & & \\
  \hline
  test--\ref{test-} & & & & & & & & & & & & & & \\
  \hline
  test--\ref{test-} & & & & & & & & & & & & & & \\
  \hline
\end{tabular}
\caption{\bf Functional Requirements Tests Traceability} \label{tab:fr-test-traceability}
\end{table}


\begin{table} [H]
  \centering
  \begin{tabular}{|c|c|c|c|c|c|c|c|c|c|c|c|c|c|c|c|c|c|c|}
\hline
TestID & AD1 & AD2 & UR1 & UR2 & PR1 & PR2 & OR1 & OR2 & MR1 & MR2 & SR1 & SR2 & CR1 & CR2 & LR1 & LR2 & S1 & S2 \\
\hline
test-AD\ref{test-AD1} & $\times$ & & & & & & & & & & & & & & & & & \\
\hline
test-AD\ref{test-AD2} & & $\times$ & & & & & & & & & & & & & & & & \\
\hline
test-UR\ref{test-UR1} & & & $\times$ & & & & & & & & & & & & & & & \\
\hline
test-UR\ref{test-UR2} & & & & $\times$ & & & & & & & & & & & & & & \\
\hline
test-PR\ref{test-PR1} & & & & & $\times$ & & & & & & & & & & & & & \\
\hline
test-PR\ref{test-PR2} & & & & & & $\times$ & & & & & & & & & & & & \\
\hline
test-OR\ref{test-OR1} & & & & & & & $\times$ & & & & & & & & & & & \\
\hline
test-OR\ref{test-OR2} & & & & & & & & $\times$ & & & & & & & & & & \\
\hline
test-MR\ref{test-MR1} & & & & & & & & & $\times$ & & & & & & & & & \\
\hline
test-MR\ref{test-MR2} & & & & & & & & & & $\times$ & & & & & & & & \\
\hline
test-SR\ref{test-SR1} & & & & & & & & & & & $\times$ & & & & & & & \\
\hline
test-SR\ref{test-SR2} & & & & & & & & & & & & $\times$ & & & & & & \\
\hline
test-CR\ref{test-CR1} & & & & & & & & & & & & & $\times$ & & & & & \\
\hline
test-CR\ref{test-CR2} & & & & & & & & & & & & & & $\times$ & & & & \\
\hline
test-LR\ref{test-LR1} & & & & & & & & & & & & & & & $\times$ & & & \\
\hline
test-LR\ref{test-LR2} & & & & & & & & & & & & & & & & $\times$ & & \\
\hline
test-S\ref{test-S1} & & & & & & & & & & & & & & & & & $\times$ & \\
\hline
test-S\ref{test-S2} & & & & & & & & & & & & & & & & & & $\times$ \\
\hline
\end{tabular}
\caption{Non-Functional Requirements Tests Traceability} \label{tab:nfr-test-traceability}
\end{table}


\begin{table} [H]
  \centering
  \begin{tabular}{|c|c|c|c|c|c|c|c|c|c|}
  \hline
  Test ID & AC1 & AC2 & IR1 & IR2 & IR3 & IR4 & IR5 & IR6 & IR7 \\
  \hline
  test-AC1-\ref{test-AC1-1} & $\times$ & & & & & & & & \\
  \hline
  test-AC-\ref{test-AC1-2} & $\times$ & & & & & & & & \\
  \hline
  test-AC-\ref{test-AC2-1} & & $\times$ & & & & & & & \\
  \hline
  test-IR-\ref{test-IR1-1} & & & $\times$ & & & & & &  \\
  \hline
  test-IR-\ref{test-IR2-1}  & & & & $\times$ & & & & & \\
  \hline
  test-IR-\ref{test-IR3-1}  & & & & & $\times$ & & & & \\
  \hline
  test-IR-\ref{test-IR4-1}  & & & & & & $\times$ & & & \\
  \hline
  test--\ref{test-}  & & & & & & & $\times$ & & \\
  \hline
  test-IR-\ref{test-IR6-1}  & & & & & & & & $\times$ & \\
  \hline
  test--\ref{test-}  & & & & & & & & & $\times$ \\
  \hline
\end{tabular}
\caption{\bf Safety and Security Requirements Traceability} \label{tab:sns-test-traceability}
\end{table}

\end{landscape}
\newpage




\subsection{Unit Testing Scope}

\wss{What modules are outside of the scope.  If there are modules that are
  developed by someone else, then you would say here if you aren't planning on
  verifying them.  There may also be modules that are part of your software, but
  have a lower priority for verification than others.  If this is the case,
  explain your rationale for the ranking of module importance.}

\subsection{Tests for Functional Requirements}

\wss{Most of the verification will be through automated unit testing.  If
  appropriate specific modules can be verified by a non-testing based
  technique.  That can also be documented in this section.}

\subsubsection{Module 1}

\wss{Include a blurb here to explain why the subsections below cover the module.
  References to the MIS would be good.  You will want tests from a black box
  perspective and from a white box perspective.  Explain to the reader how the
  tests were selected.}

\begin{enumerate}

\item{test-id1\\}

Type: \wss{Functional, Dynamic, Manual, Automatic, Static etc. Most will
  be automatic}
					
Initial State: 
					
Input: 
					
Output: \wss{The expected result for the given inputs}

Test Case Derivation: \wss{Justify the expected value given in the Output field}

How test will be performed: 
					
\item{test-id2\\}

Type: \wss{Functional, Dynamic, Manual, Automatic, Static etc. Most will
  be automatic}
					
Initial State: 
					
Input: 
					
Output: \wss{The expected result for the given inputs}

Test Case Derivation: \wss{Justify the expected value given in the Output field}

How test will be performed: 

\item{...\\}
    
\end{enumerate}

\subsubsection{Module 2}

...
				
\bibliographystyle{plainnat}

\bibliography{../../refs/References}

\newpage

\section{Appendix}

This is where you can place additional information.

\subsection{Symbolic Parameters}

The definition of the test cases will call for SYMBOLIC\_CONSTANTS.
Their values are defined in this section for easy maintenance.

\subsection{Usability Survey Questions?}

\wss{This is a section that would be appropriate for some projects.}

\newpage{}
\section*{Appendix --- Reflection}

\wss{This section is not required for CAS 741}

The information in this section will be used to evaluate the team members on the
graduate attribute of Lifelong Learning.

The purpose of reflection questions is to give you a chance to assess your own
learning and that of your group as a whole, and to find ways to improve in the
future. Reflection is an important part of the learning process.  Reflection is
also an essential component of a successful software development process.  

Reflections are most interesting and useful when they're honest, even if the
stories they tell are imperfect. You will be marked based on your depth of
thought and analysis, and not based on the content of the reflections
themselves. Thus, for full marks we encourage you to answer openly and honestly
and to avoid simply writing ``what you think the evaluator wants to hear.''

Please answer the following questions.  Some questions can be answered on the
team level, but where appropriate, each team member should write their own
response:


\begin{enumerate}
  \item What went well while writing this deliverable? 
  \item What pain points did you experience during this deliverable, and how
    did you resolve them?
  \item What knowledge and skills will the team collectively need to acquire to
  successfully complete the verification and validation of your project?
  Examples of possible knowledge and skills include dynamic testing knowledge,
  static testing knowledge, specific tool usage, Valgrind etc.  You should look to
  identify at least one item for each team member.
  \item For each of the knowledge areas and skills identified in the previous
  question, what are at least two approaches to acquiring the knowledge or
  mastering the skill?  Of the identified approaches, which will each team
  member pursue, and why did they make this choice?
\end{enumerate}

\end{document}